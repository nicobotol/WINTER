\section{Conclusions and future works}\label{sec:c_conclusions}

Wind is nowadays one of the most relevant renewable energy resources, and in the future it will play even a more central role to fulfill the global energy requirements.\\
The technological development trends highlight an upscaling of the size and the rated powers of the wind turbines (WTs), both for deployment onshore and offshore. While onshore installations are still more common, offshore ones offer some advantages in terms of wind availability and maximum turbine sizes, but at the same time present challenges in the commissioning, maintenance and grid interconnection. The growing dimensions rises also the costs and the loads that the turbine have to withstand hence the control for producing the maximum amount of energy while ensuring safe operations becomes a viable part to make the technology effective. To answer these needs, digital emulators are nowadays fundamental tools to accelerate the design and development procedure, enabling to test the devices without the necessity of prototypes and furthermore allowing to exploit new control features. 

The first part of the thesis is more qualitative. Initially an introduction on the WT field is given by means of a short historical evolution, the current employment, and the future development of the resource, mainly focusing on the European area because it is where the writer lives. After that, a description of the functions of the most important physical and control parts is given. Furthermore, a table comparing the drivetrain and electrical generator solutions adopted by some manufacturers is reported, in order to justify the choice done in the second part of the work. Then the picture is completed by listing the most common power electronic configurations for connecting the generator to the grid and the hierarchical levels of the control system.   

The second part of the thesis is more quantitative than the first one and its contribution is twofold since it includes the modelling and testing of a reference WT. \\
The first contribution is the development of a software simulator in the Matlab/Simulink environment. The model covers the aerodynamical wind-rotor interaction by means of a static \acrfull{BEM} algorithm, the electrical generator, and the blade pitching mechanism. Even though the model is quite general and may be used for different WTs and generators, here the DTU 10 MW reference turbine connected to a \acrfull{PMSM} has been considered, due to the wide availability of public reference data for benchmarking the results. This turbine is a variable speed, variable pitch, direct driven, offshore deployable WT, but neither the mooring system nor the hydrodynamic loads have been considered. Going more into details, the rotor inertia, the static map for the \acrfull{cp}, the static map for the pitch angle are firstly computed starting from blade's aerodynamic characteristics, and then validated on the corresponding literature data. Whenever reference data were not directly available for the DTU 10 MW, the computational procedure has been firstly validated on the NREL 5 MW and then applied to the turbine under analysis.\\
In the second developed part, the emulator has been used to test some control methods for the blade pitching mechanism and the generator. The former, is used here for limiting the extracted power at the rated values. This feature is the simplest that may be implemented, while more advanced controllers may be also used for limiting the aerodynamic loads on the blade and to provide ancillary services to the grid. The implementation of the pitch controller is done by means a \acrfull{PI} controller using the rotational speed as input and scheduling the gains with the pitch angle. In order to compute these gains, a modified version of the BEM code (i.e. the so called \textit{Frozen Wake BEM}) has been developed, validated on the NREL 5 MW WT, and later on used in the DTU 10 MW.\\
For what regards the generator, three high level control methods have been proposed for realizing the \acrfull{MPPT} in the below rated power region, and a low level \acrfull{PID} voltage control has been tuned. \\
The first high level controller aims to maximize the mechanical power extracted at the rotor side, and it is generally implemented in variable speed WTs. Differently from what is frequently done in the literature, the transmission damping term is here included in the control scheme, with a beneficial effect in driving the turbine to work in the desired point.\\
Then it is observed that the maximization of the harvested mechanical power not always corresponds to the maximization of the produced electrical ones because, with the increase of the incoming WS, some losses in the electrical generator grow faster than the increase of the mechanical power extracted from the resource. In order to maximize the electrical generator power it is necessary to include the generator losses in the expression of the power output and write a maximization problem. The obtained control method explicitly depends on the incoming WS but, in order to simplify the implementation on a WT where WS measurements are not available, an approximated solution based on averaging the control for different WS has been proposed.\\
The third method tries to take into account the uncertainties in the physical parameters (for example due to their poor identification or their unknown variations over time) by means of an \acrfull{IMM} estimator, running a bank of \acrfull{EKF} in parallel, each of which with an unique plausible model of the plant, and then combining the filters output according to their likelihood. The proposed method provides an estimation of the state of the system (i.e. made by the rotor rotational speed and the incoming WS) and also offers a solution for adapting the controller to the estimated changing in time parameters, ensuring also in this condition the MPPT.

Ultimately, the developed emulator has been tested under different WS conditions and also changing the properties of some of the actuators and controllers. \\
The results show that the emulator behaves similarly to the reference data when the standard mechanical power maximization algorithm is employed. It is also shown that the inclusion of the damping term in the high level controller improves turbine's ability to follow the static power curve. Furthermore, different pitching mechanism and values for the gains on the controllers are tested showing the flexibility of the emulator to testing multiple configurations.\\
The second algorithm shows its partial effectiveness meaning that the results are as expected in the WS regions where the average control is close to the WS dependent, while they are worse when the approximation is more rough. On the other hand, even though the increment of power is limited (i.e. below 1\%) the proposed analysis may be further expanded by introducing other losses, such as the ones in the iron and in the converters, leading to a more significant increase of power.\\
Finally, the IMM method shows promising results in tracking the time evolutions of the variable parameters. Furthermore, the extraction of more electrical energy than the case with constant gain suggests that the relative simple proposed gain adaptation may be promising. It is worth noting that the results of the last simulation should be considered with care, because of the choice of the parameters. In fact to let the IMM algorithm work properly, the variations of the physical quantity, and so the models run in the IMM algorithm, were overestimated and so the obtained results may not be be fully realistic. 

\textbf{Future works}\\
The developed emulator may be further improved under different prospective, in order to make it more realistic, flexible, and so more usefull for the study of modern WTs. The development may be focused both on the model and the control parts.

From the aerodynamic point of view, the model can be expanded by using theories more complex than BEM. Furthermore, a more realistic WS may include the spatial variability of the wind flow and the effect of the tower shadow.\\
From the mechanical standpoint, the vibrations of the blades and more complex drivetrain models may be implemented.\\
The models of the electrical part may be completed by adding further losses and non-idealities (such as the iron losses, and saturations) in the generator, and extending the conversion chain by adding the power electronic for grid connection. 

The developed controllers may run secondary objectives alongside the MPPT presented before, such as the ones for load and vibration reduction, and the individual blade pitching. \\
For what concerns the second and third control methods (i.e. maximization of generator power and IMM), on top of the assumptions and quantitative results of the conducted tests, it is important to highlight how the proposed analyses represent some valuable methods for expanding the standard MPPT control scheme based only on the maximization of the mechanical power extracted from the resource and considering constant values for the physical parameters. In further development, the same methodology presented here may be coupled with the more detailed physical models (such as the generator with more losses and the power electronic) and may considers the variation of more parameters (as the ones of the blades), also having implications on the aerodynamic. 