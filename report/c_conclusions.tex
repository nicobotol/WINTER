\section{Conclusions}\label{sec:c_conclusions}

% \begin{itemize}
%   \item Wind turbines are valuable converters to fulfill the energy requirements
%   \item Highlight the most important concepts for the state of the art.
%   \item In the thesis first of all an emulator was studied and developed in Simulink. Focus on a MW turbine with PMSM because the upscaling trend.
%   \item In the second part three control strategies were tested on the emulator. The first maximizes the power to the rotor side, the second to the generator adn the third takes into account the uncertainty in the parameters.
%   \item The max power at the rotor side requires only a wind speed measurement
%   \item The maximization of the power at the generator side is possible by including the generator losses in the conversion chain and maximizing the cost function. From the results it is seen that this control method is effective in principle but even better results may be obtained by scheduling the control according to the wind speed. Possible back compatibility.
%   \item The IMM methods seems to be promising but the results suffer from limitations in the modeling, so it needs further investigations.
% \end{itemize}

%From what concerns the mechanical transmission, direct drive and geared solutions are both currently frequently employed by manufacturers, and their choice is often related to the chosen generator which is often a \acrfull{DFIG} or \acrfull{PMSM}.
Wind is nowadays one of the most relevant renewable energy resources, and in the future it will play even a more central role to fulfill the global energy requirements.\\
The technological development trends highlight an upscaling of the size and the rated powers of the wind turbines (WTs), both for deployment onshore and offshore. While onshore installations are site is still more common, offshore ones offer some advantages in terms of wind availability and maximum turbine sizes, but at the same time present challenges in the commissioning, maintenance and grid interconnection. The growing dimensions rises also the costs and the loads that the turbine have to withstand hence the control for producing the maximum amount of energy while ensuring safe operations becomes a viable part to make the technology effective. To answer these needs, digital emulators are nowadays fundamental tools to accelerate the design and development procedure, enabling to test the devices without the necessity of prototypes and furthermore allowing to exploit new control features. \\
The contribution of the thesis is twofold. In the first part, a software simulator in the Matlab/Simulink environment is implemented. The model covers the aerodynamical wind-rotor interaction by means of a static \acrfull{BEM} algorithm, the electrical generator, and the blade pitching mechanism. Even though the model is quite general and may be used for different WTs and generators, here the DTU 10 MW reference turbine connected to a \acrfull{PMSM} has been considered, due to the wide availability of public reference data for benchmarking the results. This turbine is a variable speed, variable pitch, direct driven, offshore deployable WT, but neither the mooring system nor the hydrodynamic loads have been considered.\\
In the second part of the work, the emulator has been used to test some control methods for the blade pitching mechanism and the generator. The former, is used here for limiting the extracted power at the rated values. This feature is the simplest that may be implemented, while more advanced controllers may be also used for limiting the aerodynamic loads on the blade and to provide ancillary services to the grid. For what regards the generator, three control methods have been proposed for realizing the \acrfull{MPPT} in the below rated power region. The first one aims to maximize the mechanical power extracted at the rotor side, and it is generally implemented in variable speed WTs. Then it is observed that the maximization of the harvested mechanical power not always corresponds to the maximization of the produced electrical ones because, with the increase of the incoming WS, some losses in the electrical generator grow faster than the increase of the mechanical power extracted from the resource. In order to maximize the electrical generator power it is necessary to include the generator losses in the expression of the power output and write a maximization problem. The obtained control method explicitly depends on the incoming WS but, in order to simplify the implementation on a WT where WS measurements are not available, an approximated solution based on averaging the control for different WS has been proposed. The third method tries to take into account the uncertainties in the physical parameters (for example due to their poor identification or their unknown variations over time) by means of the \acrfull{IMM} estimator. 

The results show that the emulator behaves similarly to the reference data when the standard mechanical power maximization algorithm is employed. The second algorithm shows its partial effectiveness meaning that the results are as expected in the WS regions where the average control is close to the WS dependent, while they are worse when the approximation is more rough. On the other hand, even tough the increment of power is limited (i.e. below 1\%) the proposed analysis may be further expanded by introducing other losses, such as in the iron and in the converters, where the increase of power would be more significant. Finally, the IMM method shows promising results in tracking the time evolutions of the variable parameters and extracting more electrical energy than the case with constant gain. It is worth noting that the results of the last simulation should be considered with care, because of the choice of the parameters. In fact to let the IMM algorithm work properly, the variations of the physical quantity, and so the models run in the IMM algorithm, were overestimated and so the obtained results may not be be fully realistic. 