\newpage
\section{Case study definition}\label{sec:c_10MW_OWT}
Wind turbines are devices covered by patents and the manufactures are interested in protecting their intellectual property over them, and so technical characteristics of commercial \acrshort{WTs} are not easy to be found. This is a problem for persons, institutions and companies involved in the wind turbine development because it reduces the amount of shared data. However there are some institutions around the world that have devised the so-called \textit{Reference turbines}, which are turbines not intended to be really built but are still useful as benchmark for the comparison of results. Among these turbines, in recent years the three most famous ones have been developed by the \acrfull{NREL} and the \acrfull{DTU}, known as NREL 5 MW (2008) \cite{NREL_5MW_reference}, DTU 10 MW (2012) \cite{DTU_Wind_Energy_Report-I-0092}, and NREL 15 MW (2020) \cite{NREL_15MW} respectively. It must be noted that these turbines have higher rated power than the ones commercially available at the time of their presentation, since in the 2008 the rated power was 1.5 MW \cite{Wind_Energy_Technology_Current_Status_and_RD_Futur}, in 2012 around 5.5-6 MW \cite{European_offshore_statistics_2012}, while in 2020 the General Electric 12-MW Haliade-X was not launched in the market yet. \\
In this section the DTU 10 MW Offshore Wind Turbine is presented in terms of its main characteristics and its use in the thesis. Some of the parameters have been taken directly from the literature while others have been computed from scratch and then validate with the corresponding literature where available. The scope of this analysis is then to implement a simulator in the Matlab-Simulink environment, which will be later used for understanding the behavior of the WT.

\subsection{Characteristics of the DTU 10 MW Reference WT}
The DTU 10 MW reference WT has the following characteristics reported in \autoref{tab:DTU_10_struct}.
\begin{table}[htb]
    \caption{Structural parameters of the DTU 10 MW \acrshort{WRT}}
    \centering
    \begin{tabular}{lcc}
    \toprule
    Number of blades & 3 & \\
    Rotor orientation & Clockwise rotation - Upwind & \\
    Control & Variable Speed, Collective Pitch & \\
    Cut in wind speed & 4 & $\left[\si{\meter \per \second}\right]$ \\
    Cut out wind speed & 25 & $\left[\si{\meter \per \second}\right]$ \\
    Rotor diameter & 178.3 & $\left[\si{\meter}\right]$\\
    Hub Height & 119 & $\mesunt{\meter}$\\
    Reference air density & 1.225 & $\left[\si{\kilo\gram\per\cubic\meter}\right]$\\
    \bottomrule
    \end{tabular}
    \label{tab:DTU_10_struct}
\end{table}

\subsection{Rotor inertia}
The value of the rotor inertia is not declared in the report, and so it has been computed starting from the structural data reported in \cite{DTU_Wind_Energy_Report-I-0092}. In particular:
\begin{gather}
    \Delta m_i=\frac{\left(\xi_{i+1} + \xi_i\right)\left(r_{i+1} - r_i\right)}{2} \ \ \left[\si{\kilo\gram}\right] \\
    m_{blade} = \sum_{i=1}^{N-1}\Delta m_i \ \ \left[\si{\kilo\gram}\right]\\
    I_{blade}=\sum_{i=1}^{N-1}\Delta m_i\left(\frac{r_{i+1} + r_i}{2}\right)^2 \ \ \left[\si{\kilo\gram\square\meter}\right]\\
    I_{rotor}=3I_{blade} \ \ \left[\si{\kilo\gram\square\meter}\right]
\end{gather}
\acrshort{xi} in $ \left[\si{\kilo\gram\per\meter}\right]$ is the linear mass density at section $i-th$, $r_i$ is the distance of section $i-th$ from the rotor axes, \acrshort{N} is the number of cross-sections for which data are available.\\
Before being applied on the \acrshort{WT} under study, the method is validated on the NREL 5 MW. The results of the validation and the calculation of the inertia are reported in \autoref{tab:rotor_inertia} alongside the percentage error computed as $\text{Error}=\frac{ |\text{Computed} - \text{Reference}| }{\text{Reference}}\cdot 100$.

\begin{table}[htb]
 \caption{Validation of the rotor inertia and mass for the NREL 5MW and results for the DTU 10 MW}
\centering
	\begin{tabular}{cccc}
		\toprule
    & & Blade mass $\left[\si{\kilo\gram}\right]$ & Blade moment of inertia $\left[\si{\kilo\gram\square\meter}\right]$\\ \midrule
    \multirow{3}{*}{NREL 5 MW}		     &  Computed 	& 16845	&  1.2267e7\\
      & Reference	\cite{NREL_5MW_reference} & 17740 &  1.1776e7\\ 
      & Error & 5\% &4\% \\ \midrule
    \multirow{3}{*}{DTU 10 MW}& Computed    &  43388 & 5.2056e7\\
& Reference \cite{review_of_scaling_low} & 41700 & -\\
& Error & 4\% & -  \\  
		\bottomrule
	\end{tabular}
 \label{tab:rotor_inertia}
\end{table}

\subsection{Power coefficient and other rated parameters}\label{subsec:lookup_cp}
Even tough some values for the aerodynamic parameters are reported \cite{DTU_Wind_Energy_Report-I-0092}, they are here recomputed starting from the airfoil characteristics, mainly as a validation of the literature results and to have control on the assumptions under which the vales have been obtained. Among these, the first to be identified is the power coefficient $c_{P, opt}(\lambda_{opt}, \theta_{opt})$ that produces the maximum power extraction. To do so, a parametric study is carried out and the results are saved in a lookup table that will be used later on during the simulation: a range of values for $\lambda$ and $\theta$ are distributed between suitable bounds ($\lambda \in \left[0, 18\right]$  and $\theta \in \left[-15, 90\right] \left[\si{\degree}\right]$), and then the \acrshort{BEM} code is run for all their combinations in order to compute a mesh of \acrshort{cp} and \acrshort{ct}. The results are the contour plots reported in \autoref{fig:contour_plot}, for a reduced set of the parameters around the optimal values. The optimal \acrlong{cp} is the highest among the computed ones, and the corresponding \acrshort{TSR} and pitch angle are the optimal ($\theta=0\si{\degree}$ and $\lambda=7.91$). 

\begin{figure}[htb]
    \centering
    \begin{subfigure}{0.49\textwidth}
    \centering
    \includegraphics[width=\textwidth]{images/vectorial/contour_plot_cP.eps}
    \caption{Power coefficient}
    \label{fig:contour_cp}
    \end{subfigure}
    %\hfill
    \begin{subfigure}{0.49\textwidth}
    \centering
    \includegraphics[width=\textwidth]{images/vectorial/contour_plot_cT.eps}
    \caption{Thrust coefficient}
    \label{fig:contour_cT}
    \end{subfigure}
    \caption{Contour plot of the power coefficient $c_P$ and  the thrust coefficient $c_T$ as function of the \acrlong{TSR} $\lambda$ and the pitch angle $\theta_p$ for the DTU 10 MW}
    \label{fig:contour_plot}
\end{figure}

Given the optimal power coefficient it is possible to compute the rated wind speed $V_{0,opt}$, which is the lowest velocity at which the reference power is produced.
\begin{equation}
    V_{0, rated} = \sqrt[3]{\frac{2 \ P_{R,rated}}{A \rho c_{P, max}}} \ \ \left[\si{\meter \per \second}\right]
\end{equation}

The wind turbines are usually identified by their electrical output power and not by their incoming mechanical one, so even tough the studied turbine is called \textit{10 MW} its rated $P_{R}$ is slightly higher. For this reason a reference value of $P_{R,rated} = 10.64 \left[\si{\mega \watt}\right]$ is considered during the computations, as reported in \cite{DTU_Wind_Energy_Report-I-0092}. In order to compare the different results, the analysis are repeated twice, one for each of the two different $P_{R,rated}$: in \textit{Case 1} $P_{R,rated}=10 \ \left[\si{\mega \watt}\right]$ while in \textit{Case 2} $P_{R,rated}=10.64 \ \left[\si{\mega \watt}\right]$ and so the other values are recomputed by consequence. Furthermore, the \acrshort{TSR} from the DTU report is assumed to be 7.5 as design choice, while in the two computed cases they are the ones generating the maximum power coefficients, and are slightly higher (7.93 in the first case and 7.91 in the second). \\ 
The results are reported in \autoref{tab:DTU_10_aero} and \autoref{fig:fig_cP_cT_comp}, alongside with the reference values from the DTU report. They are quite consistent and differences may be due to the reduced number of cross-sections along the blade taken into account in the implementation of the \acrshort{BEM} algorithm and in the choice of the discretization of the intervals for \acrshort{TSR} and $\theta$. \\
In \autoref{fig:fig_cP_cT_comp} it is possible to see a great difference between the results below 7 $\left[\si{\meter \per \second}\right]$. To understand this behavior, a further explanation about the vibration of the \acrshort{WT} is required. In general, the rotation of the rotor excites the tower at frequencies multiple of 3$\mathbb{P}$, with 
\begin{equation}
  \mathbb{P}=\frac{\omega_R \ \left[\text{rpm}\right]}{60 \ \left[\si{\second \per \minute}\right]} \ \left[\si{\hertz}\right]
\end{equation} 
and $\omega_R$ the rotational speed of the \acrshort{WT}, \cite{Olimpo_Anaya‐Lara}. In order to prevent the resonance between this periodic load and the tower lowest natural frequency, which in the case described in \cite{DTU_Wind_Energy_Report-I-0092} is reported to be 0.25 $\si{\hertz}$, a minimum rotational speed of 6 $\left[\text{rpm}\right]$ is imposed to excite at least at:
\begin{equation}
    3\mathbb{P} = 3\frac{6}{60}=0.3 \ \ \mesunt{\hertz}
\end{equation}

Since below 7 $\si{\meter\per\second}$ the control algorithm would have led the turbine to work below this minimum rotational speed, then the rated $c_P$ cannot be obtained, and so the results of \autoref{fig:fig_cP_cT_comp} are explained. A further implication of this choice regarding the imposed pitch angle will be discussed later.

\begin{table}[htb]
    \caption{Aerodynamical parameters of the DTU 10 MW \acrshort{WRT} as read from the report and from the computations made}
    \centering
    \begin{tabular}{lccc}
    \toprule
    Parameter & DTU report & Case 1 & Case 2\\ \midrule
    Rated power $\left[\si{\mega \watt}\right]$ & 10 & 10 & 10.64  \\
    Rated wind speed $\left[\si{\meter \per \second}\right]$ & 11.4 & 11.2 & 11.4 \\
    Rated rotational speed $\left[\si{\radian \per \second}\right]$ & 1.005 & 0.997 & 1.014 \\
    Rated rotational speed $\left[rpm\right]$ & 9.6 & 9.51 & 9.69\\
    Optimum \acrshort{TSR} & 7.5 & 7.93 & 7.91\\
    Optimum $\theta \left[\si{\degree}\right]$ & 0 & 0 & 0 \\
    Max. power coefficient & 0.476 & 0.465 & 0.465 \\
    \bottomrule
    \end{tabular}
    \label{tab:DTU_10_aero}
\end{table}

\begin{figure}[htb]
    \centering
    \includegraphics[width=0.8\textwidth]{images/vectorial/fig_cP_cT_comp.eps}
    \caption{Comparison of the computed power and thrust coefficients with the one reported in \cite{DTU_Wind_Energy_Report-I-0092}}
    \label{fig:fig_cP_cT_comp}
\end{figure}
For the rest of the simulation, the parameters obtained in the \textit{case 2} will be used because it is based on the mechanical rated power expected from the turbine and uses the \acrlong{TSR} maximizing the \acrlong{cp} according to the computations done by me.

A further validation of the results is shown in \autoref{fig:fig_cP_cT_comp_polimi}, where a parametric plot of the coefficients as function of the \acrshort{TSR} is proposed for some values of the pitch angle (dashed lines) and compared with the ones proposed in  \cite{Variable-speed_Variable-pitch_control_for_a_wind_t} (circle markers).
\begin{figure}[htb]
    \centering
    \includegraphics[width=0.8\textwidth]{images/vectorial/fig_cP_cT_comp_polimi.eps}
    \caption{Comparison of the computed power and thrust coefficients (dashed lines) with the one reported in \cite{Variable-speed_Variable-pitch_control_for_a_wind_t} (circle markers)}
    \label{fig:fig_cP_cT_comp_polimi}
\end{figure}

The knowledge of the rated values is important because they define the working conditions allowing the harvesting of the maximum power: in other words to be able to produce the maximum power it is sufficient to ensure to work in the optimum condition.

\subsection{Static power curve}\label{subsec:static_power_curve}
The results of the \acrshort{BEM} algorithm can be used to build the static power curve reported in \autoref{fig:fig_static_power_curve}. This curve will be further expanded for the same steady state conditions in \autoref{subsec:genertaor_power_curve}.
\begin{figure}
  \centering
  \includegraphics[width=0.65\columnwidth]{images/vectorial/static_power_curve.eps}
  \caption{Mechanical power curve as function of pitch angle}
  \label{fig:fig_static_power_curve}
\end{figure}


\subsection{Pitch angle for different wind speeds}\label{subsec:pitch_map}
Since the turbine is pitch regulated, the map between the \acrshort{WS} and the corresponding pitch angle needs to be reconstructed in order to verify the static performance of the simulator. The process to build this map is straightforward. The rotor speed is kept constant at the maximum value reached in the below rated condition, then for each velocity up to the cut-out one:
\begin{enumerate}
    \item  the actual \acrshort{TSR} $\lambda_V$ is computed;
    \item a vector of possible pitch angles is generated by spacing them between an upper and lower limit;
    \item for each pitch angle $\theta_i$ the corresponding $c_{P,o}(\lambda_V, \theta_i)$ is found in the lookup table generated in \autoref{subsec:lookup_cp} and then it is used for computing the power $P_i(c_{P,i})$;
    \item applying a numerical method, the intersection between the generated power curve and the rated one is found; the angle at which the powers are the same is the one suitable for the control. Since the power curve is convex, two angles fulfil the power equation: the lowest is the one used in the stall control method, while the highest in feather.
\end{enumerate}
The results of this algorithm are reported in \autoref{fig:pitch}. In particular \autoref{fig:fig_power_vs_pitch} shows how increasing the \acrshort{WS} also the peak of the power curve increases, but the rated power is exceeded only for \acrshort{WSs} above the rated one. Furthermore \autoref{fig:fig_pitch_vs_V0} shows the pitch angle necessary for limiting the power to the rated one in case of feathering and stall control strategies, alongside with the angles proposed in the reference report \cite{DTU_Wind_Energy_Report-I-0092}. 
\begin{figure}[htb]
    \centering
    \begin{subfigure}{0.49\textwidth}
    \centering
    \includegraphics[width=\textwidth]{images/vectorial/fig_power_vs_pitch.eps}
    \caption{Mechanical power curve as function of pitch angle}
    \label{fig:fig_power_vs_pitch}
    \end{subfigure}
    %\hfill
    \begin{subfigure}{0.49\textwidth}
    \includegraphics[width=\textwidth]{images/vectorial/fig_pitch_vs_V0.eps}
    \caption{Pitch angle as function of the \acrshort{WSs}}
    \label{fig:fig_pitch_vs_V0}
    \end{subfigure}
    \caption{Influence of the pitch angle on the mechanical power curve and pitch angle required to limit the mechanical power at the rated one for different \acrshort{WSs}}
    \label{fig:pitch}
\end{figure}

Even though both fathering and stall limit the power at the same amount (see \autoref{fig:fig_power_vs_V0}) they have different implications on the thrust forces (see \autoref{fig:fig_thrust_vs_V0}) and so the former approach is usually preferred over the latter one. Since it is the more relevant method, only the feather will be taken into account in the following analysis.

\begin{figure}[htb]
    \centering
    \begin{subfigure}{0.49\textwidth}
    \centering
    \includegraphics[width=\textwidth]{images/vectorial/fig_power_vs_V0.eps}
    \caption{Mechanical power as function of wind speed}
    \label{fig:fig_power_vs_V0}
    \end{subfigure}
    %\hfill
    \begin{subfigure}{0.49\textwidth}
    \includegraphics[width=\textwidth]{images/vectorial/fig_thrust_vs_V0.eps}
    \caption{Thrust force as function of wind speed}
    \label{fig:fig_thrust_vs_V0}
    \end{subfigure}
    \caption{Relation between the \acrshort{WS}, the power and the thrust for both feather and stall pitching}
    \label{fig:P_T_vs_V0}
\end{figure}

In \autoref{fig:fig_pitch_vs_V0} it is possible to see again some differences at low \acrshort{WSs} ($4\div7 \, \si{\meter\per\second}$), which are related to the one of \autoref{subsec:lookup_cp}. Since operating at a fixed rotational speed in that range of \acrshort{WS} is not optimal because the \acrshort{TSR} is fixed at a value lower than the nominal, then the blades are pitched to obtain the highest $c_P$ allowed by the constraint on $\lambda$. Again this effect is not taken into account in the thesis and the pitch is assumed to be 0$\si{\degree}$ below rated \acrshort{WS}.


\newpage