\usepackage[utf8]{inputenc}  
\usepackage[english]{babel}
\usepackage{float}
\usepackage{amsfonts}
\usepackage{amsmath}
\usepackage{mathtools}
\usepackage{xfrac}
\usepackage{caption}
% \usepackage{mdwlist}
\usepackage{subcaption}
\usepackage{color}
\usepackage[dvipsnames, table]{xcolor}
% \usepackage{circuitikz}
\usepackage{multirow}
\usepackage{booktabs}
\usepackage{enumitem}
\usepackage{listings}
\usepackage{parskip}
\usepackage{array}
\usepackage{color}
\usepackage[titletoc,title]{appendix}
%\def\doubleunderline#1{\underline{\underline{#1}}}
\usepackage{fancyref}
%\usepackage{dsfont}
\usepackage{graphicx,wrapfig,lipsum}
%\usepackage{epstopdf}
%\usepackage{blindtext}
\usepackage{pdfpages}
\usepackage{booktabs}
\usepackage[acronym, nonumberlist]{glossaries}
% \usepackage[acronym]{glossaries}
\usepackage{geometry}
\geometry{a4paper,tmargin=3cm,bmargin=3cm, lmargin=2cm,rmargin=2cm}
%\renewcommand{\baselinestretch}{1.25}
%\usepackage{chemformula}
\usepackage{csquotes}
%\usepackage[version=4]{mhchem}
\usepackage{hyperref}
\usepackage[backend=biber,style=numeric,sorting=none]{biblatex}
\addbibresource{document.bib}
%\usepackage{mathptmx}
\usepackage{siunitx}
%\usepackage{dirtytalk}
\usepackage{tikz}
\usetikzlibrary{shapes,arrows}
\usetikzlibrary{math} %needed tikz library
\usepackage{multicol}
\usepackage{upgreek}
\usepackage{listings}
\usepackage{color}
\usepackage{svg}
\usepackage{cancel}
\usepackage{makecell} % for having multirow text in tables
\usepackage{booktabs, multirow} % for borders and merged ranges
\usepackage{soul}% for underlines
%\usepackage[table]{xcolor} % for cell colors
\usepackage{changepage,threeparttable} % for wide tables
\usepackage{algorithm} % for the pseudocode
\usepackage{algpseudocode} % for the pseudocode
\usepackage{varwidth} %for the varwidth minipage environment
\newcolumntype{M}{>{\begin{varwidth}{4cm}}c<{\end{varwidth}}} %M is for Maximal column
\usepackage{glossary-mcols}


\definecolor{mygreen}{RGB}{28,172,0} % color values Red, Green, Blue
\definecolor{mylilas}{RGB}{170,55,241}

\definecolor{dkgreen}{rgb}{0,0.6,0}
\definecolor{gray}{rgb}{0.5,0.5,0.5}
\definecolor{mauve}{rgb}{0.58,0,0.82}

\lstset{frame=tb,
  language=python,
  aboveskip=3mm,
  belowskip=3mm,
  showstringspaces=false,
  columns=flexible,
  basicstyle={\small\ttfamily},
  numbers=none,
  numberstyle=\tiny\color{gray},
  keywordstyle=\color{blue},
  commentstyle=\color{dkgreen},
  stringstyle=\color{mauve},
  breaklines=true,
  breakatwhitespace=true,
  tabsize=3
}

%\usepackage[utf8]{inputenc}
\usepackage{listingsutf8}
\lstset{language=Matlab,%
    %basicstyle=\color{red},
    breaklines=true,
    extendedchars=true,
    morekeywords={matlab2tikz},
    keywordstyle=\color{blue},%
    morekeywords=[2]{1}, keywordstyle=[2]{\color{black}},
    identifierstyle=\color{black},%
    stringstyle=\color{mylilas},
    commentstyle=\color{mygreen},%
    showstringspaces=false,%without this there will be a symbol in the places where there is a space
    numbers=left,%
    numberstyle={\tiny \color{black}},% size of the numbers
    numbersep=9pt, % this defines how far the numbers are from the text
    emph=[1]{for,end,break},emphstyle=[1]\color{blue}, %some words to emphasise
    %emph=[2]{word1,word2}, emphstyle=[2]{style},    
}

% for measurement units
\newcommand{\mesunt}[1]{\left[\si{#1}\right]}
\newcommand{\rpm}{\left[\text{rpm}\right]}

% gains of the controllers
\def\GenkpMacroMan{2.787e+00}
\def\GenkiMacroMan{1.499e+02}
\def\GenkdMacroMan{1.356e-03}
\def\GentaudOneMacroMan{6.667e-05}
\def\GenMarginMan{8.354e+01}
\def\GenkpMacroAuto{2.923e+00}
\def\GenkiMacroAuto{7.109e+02}
\def\GenkdMacroAuto{1.442e-03}
\def\GentaudOneMacroAuto{6.667e-05}
\def\GenMarginAuto{7.876e+01}
