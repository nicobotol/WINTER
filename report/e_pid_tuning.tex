\newpage
\section{PID tuning}\label{sec:e_pid_tuning}
 Given G(s), the transfer function of the generic system to be controlled, the tuning is done following this reported procedure:
 \begin{enumerate}
    \item The transfer function of the plant to be controlled is rewritten as:
    \begin{equation}
        G(s) = \frac{\prod zeros(G(s))}{\prod poles(G(s))}
    \end{equation}
    \item Poles are sorted according to their magnitude, to take into account the possible presence of complex conjugate ones. 
    \item The general PID regulator transfer function is written as:
    \begin{gather}
         R(s) = k_p+\frac{k_i}{s}+k_ds=ki\frac{(1+s\uptau_p)(1+s\uptau_d)}{s}
         \label{eq:regulator3}
    \end{gather}
    but since the high frequency response wants to be limited, then a high frequency pole is introduced
    \begin{gather}
         R(s) = \left(k_p+\frac{k_i}{s}+k_ds\right)\cdot \frac{1}{(1+s\uptau_{d1})} = ki\frac{(1+s\uptau_p)(1+s\uptau_d)}{s(1+s\uptau_{d1})}
         \label{eq:regulator4}
    \end{gather}
    \item The zeros of the regulator are used to cancel out the poles of the plant. In particular one of them is usually placed at the maximum pole (i.e. the one at the highest frequency) 
    \begin{gather}
        \uptau_p =\frac{1}{max\left\{pole\left(G\right)\right\}}
    \end{gather}
    while the position of the other may be chosen according to different objectives. If it is placed on the smallest pole (e.g. \autoref{eq:pole_case1}), then the magnitude of the transfer function decreases and the phase margin increases while if it ``shifted'' on the right, the opposite happens. In this system, the latter approach is chosen and the frequency of the second zero is placed at the logarithmic mean between the highest and the lowest ones (e.g. \autoref{eq:pole_case2} and \ref{eq:pole_case3}), while sometimes it is placed at the log mean of the two highest poles:
    \begin{gather}
        \uptau_d =\frac{1}{min\left\{pole\left(G\right)\right\}} \label{eq:pole_case1}\\
        \omega_0 = \frac{\log_{10}\left(max\left\{pole(G)\right\}\right)+\log_{10}\left(min\left\{pole(G)\right\}\right)}{2} \label{eq:pole_case2}\\
        \uptau_d = \frac{1}{10^{\omega_0 }} \label{eq:pole_case3}
    \end{gather} 
    It is worth to remember that a high phase margin produces more stability of the plant, while an increase magnitude speeds the system response but may lead to an overshoot of the response.\\
    The regulator pole is assigned at a reasonably high frequency, for example one decade above the crossover one, in particular:
    \begin{gather}        
        \uptau_{d1} =\frac{1}{10\omega_{cp}}
    \end{gather}

    \item The integral gain $k_i$ is determined by solving: 
    \begin{equation}
      \left|G(s)\cdot R \left(s\right)\right| \bigg|_{s=j\omega_{cp}}=1
    \end{equation}
    \item Finally the other two gains may be found:
    \begin{gather}
        k_p = k_i \left( \uptau_p + \uptau_d \right)  \\
        k_d = k_i\tau_d\uptau_p
    \end{gather}
 \end{enumerate}