\documentclass[11pt, a4paper]{article}
\usepackage[utf8]{inputenc}  
\usepackage[english]{babel}
\usepackage{float}
\usepackage[dvipsnames, table]{xcolor}
\usepackage{array}
\usepackage{color}
\usepackage{tikz}
\usetikzlibrary{shapes,arrows}

\begin{document}
\centering
\tikzstyle{block} = [draw, fill=white, rectangle, minimum height=2.5em, minimum width=3em]
\tikzstyle{sum} = [draw, fill=white, circle, node distance=2.0cm]
\tikzstyle{input} = [coordinate]
\tikzstyle{output} = [coordinate]
\tikzstyle{pinstyle} = [pin edge={to-,thin,black}]
\tikzset{
dot/.style = {circle, fill, minimum size=#1,
              inner sep=0pt, outer sep=0pt},
dot/.default = 4pt % size of the circle diameter 
}

\usetikzlibrary{positioning}
\makeatletter
\pgfdeclareshape{record}{
\inheritsavedanchors[from={rectangle}]
\inheritbackgroundpath[from={rectangle}]
\inheritanchorborder[from={rectangle}]
\foreach \x in {center,north east,north west,north,south,south east,south west}{
\inheritanchor[from={rectangle}]{\x}
}
\foregroundpath{
\pgfpointdiff{\northeast}{\southwest}
\pgf@xa=\pgf@x \pgf@ya=\pgf@y
\northeast
\pgfpathmoveto{\pgfpointadd{\southwest}{\pgfpoint{-0.33\pgf@xa}{-0.6\pgf@ya}}}
\pgfpathlineto{\pgfpointadd{\southwest}{\pgfpoint{-0.75\pgf@xa}{-0.6\pgf@ya}}}
\pgfpathlineto{\pgfpointadd{\northeast}{\pgfpoint{-0.75\pgf@xa}{-0.6\pgf@ya}}}
\pgfpathlineto{\pgfpointadd{\northeast}{\pgfpoint{-0.33\pgf@xa}{-0.6\pgf@ya}}}
}
}
\makeatother

\begin{tikzpicture}[auto, node distance=2.5cm,>=latex']
		\node [style=block] (44) at (-12, 6.75) {$k_p\left(\theta\right)$};
		\node [style=block] (45) at (-12, 5.25) {$k_i\left(\theta\right)$};
		\node [style=sum] (46) at (-12, 3.25) {+};
		\node [record,minimum size=1cm,fill=white!30,draw] (47) at (-13.25, 3.25){};
		\node [style=block] (48) at (-13.25, 1.5) {$\frac{\omega}{n}$};
		\node [record,minimum size=1cm,fill=white!30,draw] (49) at (-10.75, 3.25) {};
		\node [style=sum] (50) at (-8.25, 3.25) {$\times$};
		\node [style=sum, node distance =2cm] (51) at (-9.25, 1.5) {$\times$};
		\node [style=block] (52) at (-7.0, 1.5) {$\frac{1}{s}$};
		\node [style=sum] (53) at (-5.75, 3.25) {+};
		\node [record,minimum size=1cm,fill=white!30,draw] (54) at (-4.5, 3.25) {};
		\node [style=input] (57) at (-14.75, 3.25) {57};
		\node [style=input] (62) at (-14.75, 6) {62};
		\node [style=dot] (63) at (-13.25, 6) {};
		\node [style=output] (69) at (-3.25, 3.25) {69};
		\node [style=dot] (70) at (-10, 3.25) {};

		\draw [->] (48) -| (46);
		\draw [->] (49) -- (50);
		\draw [->] (47) -- (46);
		\draw [->] (50) -- (53);
		\draw [->] (53) -- (54);
		\draw [->] (54) -- node {$\theta$} (69);
		\draw [->] (51) -- (52);
		\draw [->] (52) -| (53);
		\draw [-] (62) -- node {$\theta$} (63);
		\draw [->] (63) |- (44);
		\draw [->] (63) |- (45);
		\draw [->] (57) -- node {$\omega$} (47);
		\draw [->] (45) -| (51);
		\draw [->] (70) |- (51);
		\draw [->] (44) -| (50);  
		\draw [->] (46) -- (49);  

\end{tikzpicture}


% \centering
% \tikzstyle{block} = [draw, fill=white, rectangle, 
%     minimum height=2.5em, minimum width=3em]
% \tikzstyle{sum} = [draw, fill=white, circle, node distance=1cm]
% \tikzstyle{input} = [coordinate]
% \tikzstyle{output} = [coordinate]
% \tikzstyle{pinstyle} = [pin edge={to-,thin,black}]

% \begin{tikzpicture}[auto, node distance=2.5cm,>=latex']

%     \node [input, name=pitch_input] {};
%     \node [input, name=omega_input] {};
%     \node [block, right of=pitch_input] (prop_gain) {$k_p$};
%     \node [block, below of=prop_gain] (pitch_input) {$k_i$};
%     \node [block, below of=_gain] (pitch_input) {$k_i$};
%     % \node [sum, right of=inverse_gain, node distance =2cm] (sum) {};
%     % \node [block, right of=sum] (controller) {$R_{iq}$}; % controller
%     % \node [block, right of=controller] (G_c) {$G_c$};
%     % \node [block, right of=G_c] (Yq) {$Y_{iq}$};
%     % \node [block, right of=Yq] (gain) {$\frac{3}{2}p\Lambda_{mg}$};
%     % \node [output, right of=gain] (output) {};
%     % \coordinate [below of=sum, node distance=1cm] (measurements) {};

%     % \draw [draw,->] (input) -- node {$T_G^*$} (inverse_gain);
%     % \draw [->] (inverse_gain) -- node {$I_q^*$} (sum);
%     % \draw [->] (sum) -- node {}(controller);
%     % \draw [->] (controller) -- node [name=controllerG_c] {} (G_c);
%     % \draw [->] (G_c) -- node [name=G_cYq] {} (Yq);
%     % \draw [->] (Yq) -- node [name=Yqgain] {$I_q$} (gain);
%     % \draw [->] (gain) -- node [name=gainout] {$T_G$} (output); 
%     % \draw [-] (Yqgain) |- (measurements);
%     % \draw [->] (measurements) -- node[pos=0.8] {$-$} (sum);
    
%     %\draw [->] (measurements) -| node [pos=0.99] {$-$} (sum); 
%     %\draw [->] (sum3) -- node [name=sum3sys4] {} (sys4);
%     %\draw [->] (sys4) -- node [name=sys4output] {$\Omega_m$} (output);
%     %\draw [->] (sys5) -| node [name=sys5sum2] [pos=0.99] {$-$} (sum2);
%     %\draw [->] (sys4output) |- node [near end] [name=outputsys5] {} (sys5);
%     %\draw [draw, ->] (T_L) -- node[pos=0.99] {$-$} node {$T_{areo}$} (sum3);

% \end{tikzpicture}



\end{document}