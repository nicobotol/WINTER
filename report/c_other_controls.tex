\section{Turbine control base on model-free methods}\label{sec:c_other_controls}
The control methods developed so far relays on the knowledge of the system model and the also the environmental parameters such as the air density. In fact in the calculation of the gains $K_{opt}$ and $K_{opt,GE}$ a single value of $c_P$ and $\rho$ has been used assuming that them do not change in time, for example due to aging of the system or evolution of the air's temperature and pressure. In order to take these quantities into account,  

\subsection{Comparison of the electrical power output in the case of different gains in the power controller}\label{subsec:c_different_KoptGE}
The first study done to identify the effect of a changing $K_{opt,GE}$ is to compare the electrical power output of the generator for its  different values. In particular, 5 values in the neighborhood of the optimal has been identified (second column of \autoref{tab:error_KoptGE}). Then a ramp WS profile between 4 and 12 $\mesunt{\meter\per\second}$ lasting 1000 $\mesunt{\second}$ has been generated.
\begin{table}[htb]
  \centering
  \begin{tabular}{cc|cc}
  \toprule
  Sim. & K & RMS error & Norm. RMS error\\ 
   &  $\mesunt{\newton \meter \square \second}$ & $\mesunt{\mega\watt}$ & \% $\left[-\right]$ \\ \midrule
  1 & 0.85$K_{opt,GE}$  & 1.20 & 16.01\\
  2 & 0.90$K_{opt,GE}$  & 0.75 & 11.15\\
  3 & 1.00$K_{opt,GE}$  & 0.08 & 6.90\\
  4 & 1.10$K_{opt,GE}$  & 0.12 & 7.00\\
  5 & 1.15$K_{opt,GE}$  & 0.14 & 7.13\\ \bottomrule
  \end{tabular}
  \caption{Error with the use of different $K_{opt,GE}$ values}
  \label{tab:error_KoptGE}
  \end{table}

The resulting static power curve has been reported in \autoref{fig:Kopt_GE_PGE}.
\begin{figure}
  \centering
  \includegraphics[width=0.7\columnwidth]{images/vectorial/2023_10_18_14_54_25fig_electrical_power_param_zoom.eps}
  \caption{Electrical power output of the generator for different values of $K_{opt,GE}$. The values of $K_{opt,GE}$ used in each simulation are the ones reported in \autoref{tab:error_KoptGE}.}
  \label{fig:Kopt_GE_PGE}
\end{figure}

From the graphs reported in \autoref{fig:Kopt_GE_PGE} it could be seen that the gain used in the first two simulations provides a power curve that is quite close to the third one (i.e. the optimal) in the below rated region. On the other hand, once the rated WS is reached, then the power command is below the saturation and so the rated power is not reached. The committed errors in this zone produces (16.01\% and 11.15\% respectively) are highly influenced by the above rated mismatch. \\
For what concerns the last two simulations, the error committed in the below rated region is higher (as could be seen in the figure) but the saturation of the power is reached, and so the overall error is lower than the first two cases. \\
The overall result is that the optimal gain is the one computed previously, and each deviation produces a degradation of the power curve. On the other hand, an overestimation of the gain is less worse than an underestimation.

\subsection{Interactive Multiple Model}\label{subsec:IMM}
\subsubsection{Motivation of the IMM}
\subsubsection{Steps of the implementation of the IMM}
\begin{gather}
  \hat{x}_k^- = f(\hat{x}_{k-1}^+, u_k)\\
  F_k = \frac{\partial f}{\partial x}(\hat{x}_{k-1}^+, u_k)\\
  P_k^- = F_k P_{k-1}^+ F_k^T + Q_k\\
  \hat{y}_k = h(\hat{x}_k^-, u_k)\\
  H_k = \frac{\partial h}{\partial x}(\hat{x}_k^-, u_k)\\
  \hat{x}_k^+ = \hat{x}_k^- + L_k (y_k - \hat{y}_k)\\
  L_k = P_k^- H_k^T (H_k P_k^- H_k^T + R_k)^{-1}\\
  P_k^+ = (I - L_k H_k) P_k^-\\
  S_k^{(j)} = H_k^{(j)} P_k^{(j)-} (H_k^{(j)})^T + R_k^{(j)}\\
  \Lambda_k^{(j)} = \frac{1}{\sqrt{2 \pi \lvert S_k^{(j)} \rvert}} \exp{\left(-\frac{1}{2}\left(z_k^{(j)}\right)^T \left(S_k^{(j)}\right)^{-1} z_k^{(j)}\right)}\\
  z_k^{(j)} = y_k - \hat{y}_k^{(j)}\\
  \mu_k^{(j)+} = \frac{\mu_k^{(j)-}\Lambda_k^{(j)}}{\sum_{j} \mu_k^{(j)-}\Lambda_k^{(j)}}\\
  \hat{x}_k^+ = \sum_j \mu_k^{(j)+} \hat{x}_k^{(j)+}\\
  P_k^+ = \sum_j \mu_k^{(j)+} \left(P_k^{(j)+} + \left(\hat{x}_k^+ - \hat{x}_k^{(j)+}\right)\left(\hat{x}_k^+ - \hat{x}_k^{(j)+}\right)^T\right)\\
  \mu_{k+1}^{(j)-} = \sum_i \pi_{ij}\mu_k^{(i)+}\\
  \mu_k^{(i|j)-} = \frac{\pi_{ij}\mu_k^{(i)+}}{\mu_{k+1}^{(j)-}} = \frac{\pi_{ij}\mu_k^{(i)+}}{\sum_i \pi_{ij}\mu_k^{(i)+}}\\
  \tilde{x}_k^{(j)+} = \sum_i \mu_k^{(i|j)-} \hat{x}_k^{(i)+}\\
  \tilde{P}_k^{(j)+} = \sum_i \mu_k^{(i|j)-} \left(P_k^{(i)+} + \left(\tilde{x}_k^{(j)+} - \hat{x}_k^{(i)+}\right)\left(\tilde{x}_k^{(j)+} - \hat{x}_k^{(i)+}\right)^T\right)
\end{gather}
\subsubsection{Definition of the boundaries for the variable parameters}
\subsubsection{Results of the IMM} 