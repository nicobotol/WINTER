\section{Turbine control base on model-free methods}\label{sec:c_other_controls}
The control methods developed so far relays on the knowledge of the system by means of its parameters (structural and environmental) and models. In fact in the calculation of the gains $K_{opt}$ and $K_{opt,GE}$, considered in the torque reference generation, a single deterministic constant in time value of all the different terms in \autoref{eq:T_G2} \textcolor{red}{check this reference} have been used. In order to take into account possible variabilities of the parameters, due to wrong estimation of some quantities or ageing, a different approach has to been used. One possibility is to use a model free method, such as the the extremum seeking, while another can be the \acrfull{IMM}.\\
In order to integrate this variability in the controller, it has to be target to estimate the value of the $K_{opt}$. \\
The global changes are going to increase the mean temperature, making the assumption of $\rho=1.225$ no more valid. For example the increase of temperature would lead to a change of 0.5\% in the site presented in \cite{en12112038}.

\subsection{Comparison of the electrical power output in the case of different gains in the power controller}\label{subsec:c_different_KoptGE}
The first study done to identify the effect of a changing $K_{opt,GE}$ is to compare the electrical power output of the generator for its  different values. In particular, 5 values in the neighborhood of the optimal has been identified (second column of \autoref{tab:error_KoptGE}). Then a ramp WS profile between 4 and 12 $\mesunt{\meter\per\second}$ lasting 1000 $\mesunt{\second}$ has been generated.
\begin{table}[htb]
  \centering
  \begin{tabular}{cc|cc}
  \toprule
  Sim. & K & RMS error & Norm. RMS error\\ 
   &  $\mesunt{\newton \meter \square \second}$ & $\mesunt{\mega\watt}$ & \% $\left[-\right]$ \\ \midrule
  1 & 0.85$K_{opt,GE}$  & 1.20 & 16.01\\
  2 & 0.90$K_{opt,GE}$  & 0.75 & 11.15\\
  3 & 1.00$K_{opt,GE}$  & 0.08 & 6.90\\
  4 & 1.10$K_{opt,GE}$  & 0.12 & 7.00\\
  5 & 1.15$K_{opt,GE}$  & 0.14 & 7.13\\ \bottomrule
  \end{tabular}
  \caption{Error with the use of different $K_{opt,GE}$ values}
  \label{tab:error_KoptGE}
\end{table}

The resulting static power curve has been reported in \autoref{fig:Kopt_GE_PGE}.
\begin{figure}
  \centering
  \includegraphics[width=0.7\columnwidth]{images/vectorial/2023_10_18_14_54_25fig_electrical_power_param_zoom.eps}
  \caption{Electrical power output of the generator for different values of $K_{opt,GE}$. The values of $K_{opt,GE}$ used in each simulation are the ones reported in \autoref{tab:error_KoptGE}.}
  \label{fig:Kopt_GE_PGE}
\end{figure}

From the graphs reported in \autoref{fig:Kopt_GE_PGE} it could be seen that the gain used in the first two simulations provides a power curve that is quite close to the third one (i.e. the optimal) in the below rated region. On the other hand, once the rated WS is reached, then the power command is below the saturation and so the rated power is not reached. The committed errors in this zone produces (16.01\% and 11.15\% respectively) are highly influenced by the above rated mismatch. \\
For what concerns the last two simulations, the error committed in the below rated region is higher (as could be seen in the figure) but the saturation of the power is reached, and so the overall error is lower than the first two cases. \\
The overall result is that the optimal gain is the one computed previously, and each deviation produces a degradation of the power curve. On the other hand, an overestimation of the gain is less worse than an underestimation.\\

\subsection{Interactive Multiple Model}\label{subsec:IMM}
\textcolor{red}{What does a filter is? }\\
\textcolor{red}{Properties of the EKF}\\
\textcolor{red}{Describe the conditions on the noise of the filter.}
The basic idea behind the IMM is to run a parallel bank of $N_j$ filters estimating the same quantity, each of them using a different model of the system. Later on, the overall estimation of interest is given by the sum of all the filters' output weighted by their likelihood conditioned on the measurement. \\
\subsubsection{Motivation of the IMM}
\subsubsection{Steps of the implementation of the IMM}
Here the procedure for implementing the IMM is reported, following the method proposed in \cite{kalman_based_IMM}. In particular the estimation of the quantity os given in 4 subsequently steps named filtering, mode probability updating, state combination and filter interaction. A graphical visualization of the method is given in \autoref{fig:IKK_schema}
\begin{figure}[htb]
  \centering
  \includegraphics[width=\0.5\columnwidth]{images/IMM_schema.jpg}
  \caption{Graphical visualization of a IMM, source \cite{kalman_based_IMM}}
  \label{fig:IKK_schema}
\end{figure}

\texbf{Adopted notation}\\
\tetcolor{red}{Coherence on the notation of subscript k}
\begin{gather}
  \text{State: } x = \omega_R\\
  \text{Input: } u = T_R\\
  \text{Measured quantity: } y = \omega_R \\
  \text{Model of the plant: }  x_{k+1} = f(x_k, u_k, \nu_k) = x_k + \frac{(T_{R,k} - K_{opt}\omega_k^2 - B_{eq}x_k)}{I_{eq}} + \nu_k \, \nu \thicksim \mathcal{N}(0, Q \\
  \text{Model of the measurement instrument: } y = h(x, u, \varepsilon) = \omega_R + \varepsilon \, \varepsilon \thicksim \mathcal{N}(0, W)  \\
  \text{Covariance matrix of the process: } Q = \textcolor{red}{Put a value}
  \text{Covariance matrix of the measurement: } W = \textcolor{red}{Put a value}
  \text{Covariance matrix of the state estimation: } P
  \text{Linearized model of the plant w.r.t. the states: } F = \frac{\partial f}{\partial x}(x, u) = \left[-\frac{2\,K_{opt}}{I_{eq}}x - \frac{B_{eq}}{I_{eq}}\right]\\
  \text{Linearized model of the measurement w.r.t. the states: } H = \frac{\partial h}{\partial x}(x, u) = \left[1\right]\\
  \text{Linearized model of the noise: } G = \frac{\partial f}{\partial \nu}(x, u)\\
  \text{Mode probability: } \mu^{(j)}\\
  \text{Likelihood of a filter: } \Lambda^{(j)}
\end{gather}

\texbf{Filtering}\\
The EKF is based on two step, named prediction and update which will be here presented. The apex (j) stresses the fact that EKF has to be repeated for each of the $N_j$ filters. 

Prediction step: the state dynamic and covariance are propagated based on the model only 
\begin{gather}
  \hat{x}_k^{(j)-} = f(\hat{x}_{k-1}^{(j)+}, u_k)\\
  F_k^{(j)} = \frac{\partial f}{\partial x}(\hat{x}_{k-1}^{(j)+}, u_k)\\
  G_k^{(j)} = \frac{\partial f}{\partial x}(\hat{x}_{k-1}^{(j)+}, u_k)\\
  P_k^{(j)-} = F_k^{(j)} P_{k-1}^{(j)+} \left(F_k^{(j)}\right)^T + G_k^{(j)} Q_k \left(G_k^{(j)}\right)^T
\end{gather}
Update step: the state dynamic and covariance are modified based on a measurement 
\begin{gather}
  \hat{y}_k = h(\hat{x}_k^{(j)-}, u_k, \varepsilon_k)\\
  H_k = \frac{\partial h}{\partial x}(\hat{x}_k^{(j)-}, u_k)\\
  L_k = P_k^- H_k^T (H_k P_k^- H_k^T + R_k)^{-1}\\
  \hat{x}_k^+ = \hat{x}_k^- + L_k (y_k - \hat{y}_k)\\
  P_k^+ = (I - L_k H_k) P_k^-\\
\end{gather}

\textbf{Mode probability updating}\\
In this step the mode probabilities are updated from the filter likelihood (i.e. how likely the filter provides a good state estimate from the measurement). This is done assuming that the error residuals are Gaussianly distributed.  
\begin{gather}
  \text{Residual: } z_k^{(j)} = y_k - \hat{y}_k^{(j)}\\
  \text{Uncertainty on the residual: }S_k^{(j)} = H_k^{(j)} P_k^{(j)-} (H_k^{(j)})^T + R_k^{(j)}\\
  \text{Likelihood: } \Lambda_k^{(j)} = \frac{1}{\sqrt{2 \pi \lvert S_k^{(j)} \rvert}} \exp{\left(-\frac{1}{2}\left(z_k^{(j)}\right)^T \left(S_k^{(j)}\right)^{-1} z_k^{(j)}\right)}\\
  \mu_k^{(j)+} = \frac{\mu_k^{(j)-}\Lambda_k^{(j)}}{\sum_{j} \mu_k^{(j)-}\Lambda_k^{(j)}}
\end{gather}

\textbf{State combination}\\
The output state of the IMM is computed by combining the state of each filter by the corresponding covariance:
\begin{gather}
  \hat{x}_k^+ = \sum_j \mu_k^{(j)+} \hat{x}_k^{(j)+}\\
  P_k^+ = \sum_j \mu_k^{(j)+} \left(P_k^{(j)+} + \left(\hat{x}_k^+ - \hat{x}_k^{(j)+}\right)\left(\hat{x}_k^+ - \hat{x}_k^{(j)+}\right)^T\right)
\end{gather}
\textcolor{red}{Write (probably) here how i compute the global K to apply on the feedback line}

\textbf{Filter interaction}\\
In this last step, the filter with higher probabilities modify the estimates of the one with lower probabilities. Each filter is considered as a mode, and the switching process of the modes is modelled by a time-invariant Markov chain. The mode transition probability $\pi_{ij}$ describes hoe likely the mode \textit{i} is change to mode \textit{j}, and it is a designed parameter. 
\begin{gather}
  \mu_{k+1}^{(j)-} = \sum_i \pi_{ij}\mu_k^{(i)+}\\
  \mu_k^{(i|j)-} = \frac{\pi_{ij}\mu_k^{(i)+}}{\mu_{k+1}^{(j)-}} = \frac{\pi_{ij}\mu_k^{(i)+}}{\sum_i \pi_{ij}\mu_k^{(i)+}}\\
  \tilde{x}_k^{(j)+} = \sum_i \mu_k^{(i|j)-} \hat{x}_k^{(i)+}\\
  \tilde{P}_k^{(j)+} = \sum_i \mu_k^{(i|j)-} \left(P_k^{(i)+} + \left(\tilde{x}_k^{(j)+} - \hat{x}_k^{(i)+}\right)\left(\tilde{x}_k^{(j)+} - \hat{x}_k^{(i)+}\right)^T\right)
\end{gather}
\textcolor{red}{Write the structure of $\Pi$}
\textcolor{red}{The values of the parameter given by the weighted mean of the different coefficients time their probability.}

\subsubsection{Definition of the boundaries for the variable parameters}
The IMM requires a-priori definitions of the parameters of the models among which the algorithm converges towards the most suitable one. It could be easily understood how the choice of these parameters is crucial for letting the algorithm work properly.\\ 
In principle one can set a distribution fort the wind speed and apply a Monte Carlo method for defining the distribution of all the others parameters, such as the deflection fo the blade (and so the rotor radius). A simplified method is to define a range of variation of the blade, supposing a normal distribution and then setting a threshold at the undeformed length observing that the length cannot be increased but only increased. \\
The $c_P$ is obtained by a similar approach: once a distribution for WS, rotational speed, and pitch angle are defined, the combination of the first two defines a distribution for the TSR which can be used alongside the third distribution for obtaining the $c_P$ interpolating the lookup table. It must be noted that the use of the $c_P$ obtained y means of a static analysis with different values of air density and blade length is not correct, because values different from the ones used in the generation of the table itself. 

For all the others parameters a normal distributions has been assumed, with mean on the corresponding rated/reference values and standard deviation (Std) defined with the assumptions reported in \autoref{tab:parameters_KoptGE}. It must be noted that the choice of these parameters has been done on realistic but arbitrary choices, and so possibly different values could be used.  

\begin{table}[htb]
  \centering
  \begin{tabular}{lcccc}
  \toprule
  Parameter & Mean & Std & MU & Motivation\\ \midrule
  Rotational speed $\omega$ & 0.988 & 0.0169 & $\mesunt{\radian\per\second}$ & More than 99\% distribution within 5\% rated value\\
  Air density $\rho$ &  1.225 & 0.0204 & $\mesunt{\kilo\gram\per\cubic\meter}$ & More than 99\% distribution within 5\% nominal value \\
  Rotor radius $R$  & 89.17 & 1.3333 & $\mesunt{\meter}$ & More than 99\% distribution within 4 $\mesunt{\meter}$ deformation \\
  Wind speed $V_0$  & 11.44 & 0.5719 & $\mesunt{\meter\per\second}$ & More than 99\% distribution within 15\% nominal value \\
  Pitch angle $\theta$ & 0.0 & 0.0058 & $\mesunt{\radian}$ & More than 99\% distribution within  1 $\si{\degree}$ \\
 \bottomrule
  \end{tabular}
  \caption{Error with the use of different $K_{opt,GE}$ values}
  \label{tab:parameters_KoptGE}
\end{table}

\begin{figure}[htb]
  \begin{subfigure}{0.5\columnwidth}
    \centering
    \includegraphics[width=\columnwidth]{images/vectorial/2023_10_19_15_17_27R_distribution}
    \caption{Distribution of the rotor radius}
    \label{fig:R_distribution}
  \end{subfigure}
  \begin{subfigure}{0.5\columnwidth}
    \centering
    \includegraphics[width=\columnwidth]{images/vectorial/2023_10_19_15_17_27K_GE_distribution}
    \caption{Distribution of the $K_{opt,GE}$ obtained by the Monte Carlo simulation}
    \label{fig:K_GE_distribution}
  \end{subfigure}
\end{figure}

Given the distribution of the $K_{opt,GE}$ gain it is possible to use it in order to design the IMM, by means of selecting a given number of gains between two values of the distributions such as the 1 and 99 percentile, $7.19 \, \cdot 10^7 \mesunt{\newton \meter \square \second}$ and $1.50 \cdot 10^7 \, \mesunt{\newton \meter \square \second}$ respectively.

\subsubsection{Results of the IMM} 