\section{Control based on the generator power output}\label{sec:control_P_GE}
\subsection{Proposed control law}\label{subsec:method_control_P_GE}
In this section another approach to the pitch control is proposed. 
Below rated windspeed nothing can be done further than what was already presented, since the limitation to the output power is the maximum amount of power that the blade can extract. Since this limit is given by the $c_P$, once it is maximized the most favorable condition is achieved. \\
On the other hand, above rated the driveline and generator can be taken into account for the definition of the new set of pitch angle at which the turbine has to work in order to maximize the energy extraction. The methodology to find it is similar to the one presented in \autoref{sec:c_basic_model_control}.  \\
First of all, an expression of the power at the output of the generator as function of the the WS, the rotational speed, and the pitch angle through the power coefficient has been found in \autoref{eq:P_G_out}.  Unfortunately, the wind and rotor speed contribution cannot be expressed with only one adimensional parameter, such as the tip speed ratio, but they have to be left separately. 
\begin{gather} 
  T_G = T_W + B_{eq}\omega = \frac{P_W}{\omega} + B_{eq}\omega = \frac{A c_P \rho V_0^3 c_P(\theta, V_0, \omega)}{2\omega} + B_{eq}\omega = \frac{3}{2}p\Lambda_{mg}i_q \Rightarrow \\ \notag
  \Rightarrow iq = \frac{2}{3p\Lambda_{mg}}\left(\frac{A c_P \rho V_0^3 c_P(\theta, V_0, \omega)}{2\omega} + B_{eq}\omega\right) \\ 
  P_{G, out} = P_{G, in} - R_s i_q^2 = \eta P_R - R_s i_q^2 = \\ \notag
   = \eta \frac{A \, c_P \rho V_0^3 c_P(\theta, V_0, \omega)}{2} - \frac{R_s \left(A c_P \rho V_0^3 + 2 B_{eq}\omega^2 \right)^2}{9p^2\Lambda_{mg}^2\omega^2} \label{eq:P_G_out} 
\end{gather}
Then a parametric study in the pitch angle has been done, meaning that for each wind speed, \autoref{eq:P_G_out} has been applied to a set of pitch angles and then the one limiting the power output (at the generator) at the rated value has been identified. The results are shown in \autoref{fig:fig_new_pitch_map}. It could be seen that this control leads to a lower limitation of the angle, meaning that an higher mechanical power is extracted from the resource (since the closer we are the to null pitch, then the higher the power extracted).
\begin{figure}[htb]
  \begin{subfigure}{0.5\textwidth}
    \centering
    \includegraphics[width=\textwidth]{images/vectorial/fig_generator_new_map.eps}
    \caption{Comparison of the powers}
    \label{fig:fig_generator_new_map}
  \end{subfigure}
  \begin{subfigure}{0.5\textwidth}
    \centering
    \includegraphics[width=\textwidth]{images/vectorial/fig_new_pitch_map.eps}
    \caption{Comparison of the pitch angles for feathering control}
    \label{fig:fig_new_pitch_map}
\end{subfigure}
  \caption{Power and pitch angle comparison between the control based on the aerodynamical power and the one based on the power output at the generator side}
  \label{fig:fig_blade_control_gen_side}
\end{figure}
In order to validate these strategy the controller has been implemented in the simulator. In particular, the feedback power to the generator controller (i.e. $P_G$ in \autoref{fig:d_torque_control_3}) has been connected to the output power of the generator itself rather than with the mechanical incoming form the rotor.

\subsection{Validation}\label{subsec:validation_control_P_GE}
The test has been done in steady state conditions, with a velocity ramp between 8 and 15 $\mesunt{\meter\per\second}$ lasting 150 seconds.\\
The results are reported in \autoref{fig:fig_blade_control_gen_side_validation}. \autoref{fig:2023_06_22_21_03_26fig_pitch_param} shows how the actual pitch angle follows what is predicted from the static map of \autoref{fig:fig_new_pitch_map}. As expected, since the blade are rotated less then what was required before the mechanical power extracted from the resource is slightly higher. This is also visible in \autoref{fig:2023_06_22_21_02_45fig_generator_power_check}, where it is also highlighted that the generator produced electrical power is kept constant at the rated value above the rated WS. Finally, \autoref{fig:2023_06_22_20_58_01power_check} shows how does the presence of the inertia made the power available at the input of the generator lower than the rotor one, and so in principle the controller could be less responsive due to a small lag in reaching the rated value. 
\begin{figure}[htb]
  \begin{subfigure}{0.5\textwidth}
    \centering
    \includegraphics[width=\textwidth]{images/vectorial/2023_06_22_21_03_26fig_pitch_param.eps}
    \caption{Pitch angle}
    \label{fig:2023_06_22_21_03_26fig_pitch_param}
  \end{subfigure}
  \begin{subfigure}{0.5\textwidth}
    \centering
    \includegraphics[width=\textwidth]{images/vectorial/2023_06_22_21_03_05fig_power_param.eps}
    \caption{Mechanical power at the rotor side}
    \label{fig:2023_06_22_21_03_05fig_power_param}
  \end{subfigure}
  \begin{subfigure}{0.5\textwidth}
    \centering
    \includegraphics[width=\textwidth]{images/vectorial/2023_06_22_21_02_45fig_generator_power_check.eps}
    \caption{Incoming and outgoing power at the generator side}
    \label{fig:2023_06_22_21_02_45fig_generator_power_check}
  \end{subfigure}
  \begin{subfigure}{0.5\textwidth}
    \centering
    \includegraphics[width=\textwidth]{images/vectorial/2023_06_22_20_58_01power_check.eps}
    \caption{Check of the powers}
    \label{fig:2023_06_22_20_58_01power_check}
  \end{subfigure}

  \caption{Results of the validation of the control based on the power output at the generator side}
  \label{fig:fig_blade_control_gen_side_validation}
\end{figure}