\newpage
\section{Structure of the work and limitation of the scope}\label{sec:b_introduction}
With the provisional agreement between member states reached on March 2023 and approved in September 2023, EU set at 42.5\% the target share of energy coming from renewable sources by 2030 (\cite{rauters}, \cite{EU_targets}) which means almost doubling the existing share of renewables (\cite{EU_targets}). These targets clearly show that a lot of work must be done in order to improve our ability in converting the available resources into more usable ones. Among them, the wind energy represents nowadays the second largest source with a share of 13\%, while the first and third are are solid biomass and hydropower, with a share of 41\% and 12\% respectively \cite{ren_share}. 

The scopes of this thesis are two. First of all to develop a mathematical model of a wind turbine conversion system from the aerodynamic up to the generator which then, in the second part of the work, should be used to study the effect of different control laws.\\
The simulator should be simple and flexible enough to allow the testing of different wind conditions and control polices, but at the same time it has to be structured enough to include as much subsystems and dynamics as possible. The emulator has been developed in the Matlab/Simulink environment, taking as study case the DTU 10 MW reference wind turbine \cite{DTU_Wind_Energy_Report-I-0092}. This choice is due to the necessity of cross-validate the obtained intermediate results with public available data from a multi MW device, before using the proposed emulator for deeper investigations.\\
In a wind turbine, the controllers are necessary to maximize the converted power while ensuring safety operation and structural integrity, and their main targets are the pitch angle (i.e. the angle between the blade and the wind flow) and the generator torque. In this work three different objectives for the controllers are studied. The first has been the maximization of the power extracted from the wind. Then it is observed that this condition not always implies to obtain the highest production from the generator, which instead requires to model the characteristics of the machine and the identification of a new control law. Both these solutions relay on the assumption of perfect knowledge of the physical parameters describing the system, which is something hard to be realized in practice, at least for the ageing of the different components. For this reason, the last controller tries to include the uncertainty in the system's model during the definition of the control actions.  

The work is organized as follows. \autoref{sec:c_WT_characteristics} summarizes an overview on the historical evolutions and the prospective of the resource, \autoref{sec:c_modelling_of_subsystem} describes the most important subsystems of a WT following the conversion chain from wind to grid, \autoref{sec:control_objective} discusses the operation regions alongside the objectives of the controllers, in \autoref{sec:c_10MW_OWT} the most important characteristics of the target reference turbine are computed starting from its physical parameters, \autoref{sec:c_basic_model_model} describes the functional blocks used in the Simulink simulation and their interconnection, \autoref{sec:c_basic_model_control} reports the principle of operation of generator and pitch controllers, in \autoref{sec:c_basic_model_simulation} the results of some simulations are analyzed and discussed, \autoref{sec:c_other_controls} discuss and tests a control law based on a not-perfect knowledge of the physical parameters, and finally \autoref{sec:c_conclusions} collects the comments and conclusions on the conducted analysis.

The scope of the thesis can be summarized by means of the following technical questions. How does a wind turbine generator can be modelled? How does the different subsystems (i.e. electrical machine and blades) have to be controlled in order to maximize the power converted from the wind to mechanical? How does this control have to be modified if the objective is the maximization of the electrical power output of the generator? Is the knowledge of the physical parameters of the systems necessary for controlling the turbine and achieving the target objectives? 

\subsection{Limitation of the scope}\label{subsec:limitation_of_scope}
Since the wind turbines are complex devices their study includes a lot of different aspects and domains, such as aerodynamical, mechanical, and electrical. This implies that the same system can be seen from different prospective, making the analysis very complex. In order to restrict and concentrate the study, some limitation of the scope has been done:
\begin{itemize}
  \item The aerodynamic interaction between the incoming wind and the blades has been modelled using a formulation relaying on static assumptions even with variable wind speeds.
  \item The blades have been considered as rigid, and deployed exactly on the rotor plane, so neither deflection nor vibration related issues have been taken into account.
  \item The yaw control (i.e. the one ensuring the orientation of the rotor plane with respect to the wind speed direction) has not been considered because it acts with a dynamic slower than the ones of the pitching mechanism and generator. Furthermore, this kind of control is usually relevant during the study of wind farms made by more turbines rather than when the focus is on the operating condition of a single one. 
  \item The real devices performing the actuation have not been explicitly modeled, but their presence included in the simulation by means of a time delay between the imposed control signal and the one applied to the system. These are in particular the actuators of the blade pitching mechanism and the power electronics controlling the generator. This assumption is valid because the focus of the thesis is on the generation of the reference signal that have to tracked by the different subsystems rather then on the real actuators performing the control actions.
  \item Even tough the DTU 10 MW is an offshore turbine the effects of waves have not been considered, as the type of foundation or floating platform. Furthermore, the presence of the tower shadowing the wind flow in the stream has not been included.
  \item Since the grid integration of the turbine is not considered in the thesis, the grid interface has not been studied, neither the aspects of power quality.    
\end{itemize}