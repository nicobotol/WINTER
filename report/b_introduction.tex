\newpage
\section{Structure of the work and limitation of the scope}\label{sec:b_introduction}
With the provisional agreement between member states reached on March 2023 and approved in September 2023, EU set at 42.5\% the target share of energy coming from renewable sources by 2030 (\cite{rauters}, \cite{EU_targets}) which means almost doubling the existing share of renewables (\cite{EU_targets}). These targets clearly show that a lot of work must be done in order to improve our ability in converting the available resources into more usable ones. Among them, the wind energy represents nowadays the second largest source with a share of 13\%, while the first and third are are solid biomass and hydropower, with a share of 41\% and 12\% respectively \cite{ren_share}. 

The scopes of this thesis are two. First of all to develop a mathematical model of a wind turbine conversion system from the aerodynamic up to the generator, which, then, shall be used to study the performance of different control laws.\\
The simulator should be simple and flexible enough to allow the testing of different wind conditions and control polices, but at the same time it has to be structured enough to include as much subsystems and dynamics as possible. The emulator has been developed in the Matlab/Simulink environment, taking as study case the DTU 10 MW reference WT \cite{DTU_Wind_Energy_Report-I-0092}, which has been chosen because it has a size above the average of the market but already represented by commercially available devices (\cite{wind_europe_data_2022}). Furthermore this reference turbine has a lot of public available data to be used for cross-validate the intermediate results.\\
With the upscaling trends of WT sizes it is very important to develop reliable emulators to be used in the developing process in order to reduce the design costs, since it is not possible to relay only on real prototypes. Furthermore, with the increasing of the powers, letting the turbine working in the most favorable configuration can make a great difference in terms of converted energy. On top of that, the loads that the turbine has to withstand are higher and so the risks of failures. For all these reasons, the pivoting roles of control systems are clear, in particular the ones of the pitch angle (i.e. the angle between the blade and the wind flow) and generator torque.\\
 In this work three different objectives for the controllers are studied. The first has been the maximization of the mechanical power extracted from the wind. Then it is observed that this condition not always leads to maximum electric power production from the generator, which instead requires consideration of the characteristics of the machine and the identification of a tailored control law. Both these solutions rely on the assumption of perfect knowledge of the physical parameters describing the system, which is something hard to achieve in practice, e.g. because of the ageing of the different components. For this reason, the last controller tries to include the uncertainty in the system's model during the definition of the control actions.  

The work is organized as follows. \autoref{sec:c_state_of_the_art} reports the state of the art and some background on the topic, presenting an overview of the historical evolutions and the prospective of the resource, describes the most important subsystems of a WT following the conversion chain from wind to grid, and discusses the operation regions alongside the objectives of the controllers, \autoref{sec:c_10MW_OWT} introduces the case study used as reference for the following analysis, \autoref{sec:c_basic_model_model} describes the emulator structure for the Simulink simulation and their interconnection, \autoref{sec:c_basic_model_control} reports the principle of operation of generator and pitch controllers, in \autoref{sec:c_basic_model_simulation} the results of some simulations are analyzed and discussed, and finally \autoref{sec:c_conclusions} collects the comments and conclusions on the conducted analysis.

The thesis ultimately aims to answer to the following questions. How does a wind turbine generator can be modelled? How does the different subsystems (i.e. electrical machine and blades) have to be controlled in order to maximize the mechanical power extracted from the wind? How does this control need to be modified to achieve maximization of the electrical power output of the generator?
How much accuracy in the identification of the physical parameters is necessary to achieve the target objectives?\textcolor{red}{Rivedere questa ultima domanda}

\subsection{Limitation of the scope}\label{subsec:limitation_of_scope}
Since the wind turbines are complex devices their study spans different aspects and domains, such as aerodynamical, mechanical, and electrical. This implies that the same system can be seen from different prospective, making the analysis very complex. In order to restrict and concentrate the study, some limitation of the scope has been done:
\begin{itemize}
  \item Lumped models have been used because they are simpler then continuous ones, especially for the aerodynamic interaction between the wind and the blades.
  \item The analysis are presented from the theoretical point of view and then implemented on an emulator taking as reference test case the DTU 10 MW reference \acrfull{WT}. Even though a lot of data are publicly available in the literature, some of them are not so and have been assumed during the model of the system. 
  \item Even tough the test case WT was originally developed for an offshore deployment, typical effects of this environment such as the hydrodynamic loads, the mooring system, and the grid interconnection related issues have not been considered. 
  \item The analysis targets a single WT, so the typical effects of wind farms have not been included in the model. 
\end{itemize}


