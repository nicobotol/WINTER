\section{Description of the most important subsystems} \label{sec:c_modelling_of_subsystem}
A general schema of a WT energy conversion chain is reported in \autoref{fig:fig_drivetrain} (source \cite{Olimpo_Anaya‐Lara}). 
\begin{figure}[htb]
  \centering
  \includegraphics[width=0.5\textwidth]{images/fig_drivetrain.jpg}
  \caption{General structure of a wind turbine drive train (source \cite{Olimpo_Anaya‐Lara})}
  \label{fig:fig_drivetrain}
\end{figure}
The incoming wind flow makes the blades rotate at a reasonably low rotational speed (let's say around 10 $\left[rpm\right]$ to give an order of magnitude). This rotation is then transmitted to the generator with or without the use of a step up gearbox, according to the requirement of the electrical machine. Finally, a power electronic converter conditions the power of the generator both controlling the torque/speed of the mechanical side and also ensuring the electrical compatibility of the waveforms on the point of connection. Agin, the function and structure of this component depends on the employed generator type.\\
Conceptually on top of the physical components there are different hierarchical levels of control, that may have different objectives according to \cite{Olimpo_Anaya‐Lara}. The highest level supervise operator commands, faults, extreme events, startup and shutdown, possibly also coordinating with other turbines in the farm. Then a lower level is primary responsible of keeping the rotational speed within limits, and secondly reducing the aerodynamic loads, structural vibrations, and provide ancillary services to the grid.  The two most important controllers for the seak of this thesis are the one regulating the inclination of the blades (called pitch) and the electrical machine.\\
In the following subsections the most important subsystems will be described in order to provide some information about their structures and functionalities. These \textit{subsystems} are not only seen from a physical point of view but also as functional blocks interconnected to provide the turbine its full capabilities.

\subsection{Basic aerodynamic model for the blade}
\subsubsection{Mechanical power and power coefficient}\label{subsec:mech_pow_and_pow_coeff}
The mechanical power that a \acrshort{WT} can extract from the wind is expressed by the well known equation:
\begin{gather}
    P = \frac{1}{2}\ A \, \rho \, c_P \, V_0^3 \ \ \left[\si{\watt}\right] \label{eq:power} \\
    V_0 = \sqrt[3]{\frac{2 \ P}{A \, \rho \, c_P}} \ \ \left[\si{\meter \per \second}\right] 
\end{gather}
where A is the rotor swept area $\left[\si{\square \meter}\right]$, $\rho$ is the air density in $\left[\si{\kilo\gram\per\cubic\meter}\right]$, and $c_P$ is the power coefficient. 

The \acrfull{cp} is defined as the ratio between the power extracted by the \acrshort{WT} divided by the available power from the source:
\begin{equation}
    c_P = \frac{P_{\text{extracted}}}{P_{\text{available}}} \ \ \left[-\right]
    \label{eq:c_P}
\end{equation}
and it is computed from the analysis of the elementary blade cross section. More in detail, the $c_P$ is a function of the \acrfull{TSR} $\lambda$ and the pitch angle $\theta$. The \acrshort{TSR} is defined as:
\begin{equation}
    \lambda = \frac{\omega R}{V_0} \ \ \left[-\right]
    \label{eq:TSR}
\end{equation}
with $\omega$ being the rotor rotational speed in $\left[\si{\radian \per \second}\right]$, R is the rotor radius in $\left[\si{\meter}\right]$, and $V_0$ is the incoming wind speed in  $\left[\si{\meter \per \second}\right]$. The pitch angle is the rotation angle of the blade around its main axis, and by adjusting it the \acrfull{AOA} of the entire blade changes, producing a modification of the harvested power. The \acrshort{AOA} is the one between the incoming wind flow and blade's chordline, denoted as $\alpha$ in \autoref{fig:velocity_triangle}. 
\begin{figure}[htb]
    \centering
    \includegraphics[width=0.5\textwidth]{images/velocity_triangle.png}
    \caption{Airfoil cross-section of chord length $c$. The Angle of attack $\alpha$ is the one between the incoming \acrshort{WS} $V_0$ and the chordline. $D$ and $L$ are the produced drag and lift forces. F and M are the total force and torque on the cross section. Source of the figure: \cite{Aerodynamics_of_wind_turbines}}
    \label{fig:velocity_triangle}
\end{figure}

Usually the power is reduced by decreasing the \acrshort{AOA} by pitching the leading edge of the blades up against the wind, but also the opposite is possible \cite{Aerodynamics_of_wind_turbines}. The first strategy is called \textit{feathering} while the second \textit{stalling}.

\subsubsection{Thrust force and thrust coefficient}
The thrust force, usually denoted as T, is the force produced by the rotor in the streamwise direction. It produces the wind speed reduction from the incoming one to the one in the wake, and it is compute as:
\begin{equation}
    T = \frac{1}{2} \, A \, \rho \, c_T \, V_0^2 \ \ \left[\si{\newton}\right]
    \label{eq:thrust_coeff}
\end{equation}
The thrust coefficient $c_T$ is obtained similarly to the power one, as it will be shown later in \autoref{subsec:BEM_algorithm}.

\subsubsection{BEM algorithm}\label{subsec:BEM_algorithm}
As reviewed in \cite{HANSEN2006285}, the most common families of methods used to study the aerodynamic of a WTs are \acrfull{BEM}, lifting line, panel and vortex, generalized actuator disc, and 2D/3D full CFD. Since they made different types of assumptions then their fidelity and computational cost change. In this work only \acrshort{BEM} method will be discussed and used while the others families are only cited to give an overview of the modelling possibilities. The choice of the BEM method has been done since it is the most simple and computationally cheap while it still provides satisfactory results once good airfoil data are available \cite{HANSEN2006285}.\\
The relationship between the \acrshort{cp}, the \acrshort{TSR} and the pitch angle is not linear, and involves the aerodynamic interaction between the incoming wind and the airfoil characteristics. Different methods are available for computing this interactions, with different levels of complexity and fidelity. A common analytical mathematical tool is the so called \acrshort{BEM}, solving the problem by providing the tangential and normal forces to the blade. The assumptions on which it is based are the absence of radial dependency of the blade-wind interaction and a constant force from the blades on the flow. The last assumption corresponds to having a rotor with an infinite number of blades, and so it is later corrected with a coefficient taking into account their finite number \cite{Aerodynamics_of_wind_turbines}.\\
At the basis of the computations done in this work there is a \acrshort{BEM} algorithm implemented in Matlab whose pseudo-code is reported in \autoref{app:BEM_code}. In particular, \autoref{alg:main_BEM} reports the main functions that loops between the different TSR, pitch, and cross sections, while \autoref{alg:detailed_BEM} reports the implementation details leading to the computation of the local (i.e. for a cross section) thrust and power coefficients. A list of the main terms used in the code is reported in \autoref{tab:BEM_code_notation}. \\


\subsection{Drivetrain}
The term \textit{drivetrain} identifies all the components of the mechanical system used to transfer the power from the rotor to the generator.
The classical components of a drivetrain are the \textit{main shaft} (i.e. the shaft connected to the rotor) which usually rotates at low speed, the \textit{gearbox} (which is not always present), and the generator. According to \cite{Olimpo_Anaya‐Lara}, in 85\% of the drivetrains a gearbox is present but this value may be biased by the older installed machines. Its primary use is to step up the rotational speed, and so to use high speed machines. According to the gearbox ratio, they may be divided between \textit{medium speed} (around 1/10) and \textit{high speed} (which can reach the range 1/90-1/120).\\
The other possibility is to have a \textit{gearless} configuration, and the direct drive is the most common, while hydraulic transmission are under development, \cite{Olimpo_Anaya‐Lara}.\\
The geared concept is the older one since it has been available since 1970s, and it amplifies the inertia of the generator by the square of the inverse of the gear ratio, reducing the torque variations sensitivity. The gear-less concept was introduced in 1991, it requires a lower number of components, eliminates transmission losses and gearbox failure. On the other hand the electrical machine has higher dimensions and weights because more space for the poles is necessary.\\
In the market it is not clear which of the two solutions will be the dominant, since both are available. \autoref{tab:transmission_review} summarises some interesting features of multi mega-watt \acrshort{WT} produced by some of the most important manufacturers currently in the market. 

\begin{table}[htp]\centering
    \caption{Review of the drivetrain characteristic of some WTs available on the market. The most used electrical machines are the \acrfull{PMSM} and the \acrfull{DFIG}.} \label{tab:transmission_review}
    \scriptsize
\begin{tabular}{lccccccc}\toprule
Manufacturer &Model &Power [MW] &Location &Year &Transmission & Transmission type & \makecell{ Electrical \\ Machine} \\\midrule
\multirow{5}{*}{\thead{Siemens \\ Gamesa}} &SG 6.6-170 &6.6 &Onshore &2021 &Gearbox & - & No data \\
&5.X platform &5.6/7.0 &Onshore &2021 & Gearbox & - & No data \\
&SG 8.0-167 DD &8.0 &Offshore &2019 &Direct Drive & - &PMSM \\
&SG 11.0-200 DD &11.0 &Offshore &2022 &Direct Drive & - &PMSM \\
&SG 14-222 DD &14.0 &Offshore &2024 &Direct Drive & - &PMSM \\ \midrule
\multirow{5}{*}{Vestas} &V117-4,2 MW &4.2 &Onshore & - & Gearbox & &PMSM \\
&V164-9.5MW &9.5 &Offshore & & Gearbox &Medium speed &PMSM \\
&V164-10MW &10.0 &Offshore & & Gearbox&Medium speed &PMSM \\
&V174-9.5MW &9.5 &Offshore & & Gearbox&Medium speed &PMSM \\
&V236-15.0MW &15.0 &Offshore & & Gearbox&3 planetary stages &PMSM \\ \midrule
General &Haliade 150-6MW &6.0 &Offshore &2016 &Direct Drive & - &PMSM \\
Electric &Haliade 150-12MW &12.0 &Offshore &2019 &Direct Drive & - &PMSM \\\midrule
\multirow{2}{*}{Nordex} &N175/6.X &6.0 &Onshore & -  & Gearbox&High speed &DFIG \\
&N163/5.X &5.0 &Onshore &  & Gearbox&High speed &DFIG \\ \midrule
\multirow{4}{*}{Mingyang} &MY2.0 &2.0 &Offshore & &Gearbox & &DFIG \\
&MYSE3.0MW &2.6/3.6 &Offshore & &Gearbox & - &PMSM \\
&MYSE4.0MW &4.0/5.0 & & &Gearbox & - &PMSM \\
&MYSE6.0MW &5.5/8.3 & & &Gearbox & - &PMSM \\
\bottomrule
\end{tabular}

\end{table}

It is possible to notice that Siemens-Gamesa and General Electric use direct drive concepts, while Vestas still prefers a geared solution even tough with the same kind of electrical machine.

\subsection{Electrical generator}
The electrical generator is one of the most important components since it has the function of converting the incoming mechanical energy into electrical one. Since the market offers a lot of different types then the choice of the most suitable for an application is a key point in order to maximize the efficiency, reducing the \acrfull{OeM} and ultimately the cost of energy \cite{Olimpo_Anaya‐Lara}. According the same source, some of the requirements from the electrical machine are high efficiency, reliability, cost effectiveness, reduced dimensions and weights, ability to operate in a wide range of torques and speeds. \\
The four most important types of electrical generators employed are the \acrfull{SCIG}, the \acrfull{DFIG}, the \acrfull{WRSM} and the \acrfull{PMSM}.
\begin{itemize}
\item The \acrshort{SCIG} has a proven technology and so it is robust and its mass production is cost effective. Unfortunately it required a fixed shaft rotational speed, which limits its operation regime. Nowadays this drawback reduces its popularity, and in fact none of the large scale WTs reported in \autoref{tab:transmission_review} uses it.
\item The \acrshort{DFIG} is a kind of induction machine in which the nominal asynchronous rotational speed may be changed in a limited range. This is possible by controlling the voltage applied to the rotor, which has three windings. This type of machine is employed nowadays in some commercial WTs.
\item In the \acrshort{WRSM} machine the synchronous field is produced by a current flowing in the rotor windings. The main drawbacks of these machines are the heat losses produced by the current generating the field. According to \cite{en15186700} only few commercial WTs use this technology and they do mainly for powers below the 5 $\mesunt{\mega\watt}$. In the same reference, only the Enercon E126 (7.5 $\mesunt{\mega\watt}$) is reported as example of \acrshort{WRSM} with more than 5 $\mesunt{\mega\watt}$.
\item As could be seen in \autoref{tab:transmission_review}, the \acrshort{PMSM} is a widely used technology nowadays. In this machines the magnetic flux is produced by some magnets mounted on the rotor, and so no current is necessary to produce it. According to the position of the magnets in the rotor, two families of \acrshort{PMSM} are defined. In particular, when the magnets are included inside the iron core, the rotor is no more symmetric from the magnetic point of view, and so the machine is called \textit{anisotropic}. On the other hand, when the magnets are mounted on the surface, the machine is called \textit{isotropic}. \\
  The synchronous machines allows the operation in a wider range of speeds and torques but have also a cost drawback since the permanent magnets are expansive. 
\end{itemize}
In the case of large offshore WTs the drawbacks of the described technologies are further highlighted since as the dimensions increases also the quantity of material and hence the costs do so. For these reasons the reliability is even more important. \\
\cite{Olimpo_Anaya‐Lara} reports some interesting trends regarding new generator technologies and concepts, here not reported because not strictly relevant for the work. 

\subsection{Power electronic converters}
As will be better seen in \autoref{sec:control_objective}, the power electronic converters are the components that allow the turbine to work in a wider regions of speeds, so in maximizing the energy harvested from the wind. \cite{Olimpo_Anaya‐Lara} reports the most important requirements of those components, among whom there are the reliability (since they are reported to be at the top of the list of the most common failures). \\
The functional principle of these converter is to transform the variable voltage/frequency output of the turbine to DC and back to AC with fixed frequency/voltage suitable for connection to the grid.\\
According to the type of electrical generator and the power rating, different constructions are possible for those generators. The wider classification can be done between low voltage ($<1 \mesunt{\kilo \volt}$) and medium voltage (1-35 $\mesunt{\kilo \volt}$). The \acrfull{VSR} controls the generator torque and speed, while the \acrfull{VSI} controls the DC bus and the grid reactive power. Typical switching frequencies for these electric components are in the range 1-3 $\mesunt{\kilo \hertz}$ \cite{Olimpo_Anaya‐Lara}. \\
Two possible arrangements of power electronic devices used for achieving the power conversion at low voltage are visible in \autoref{fig:B2B_converters}.
In particular \autoref{fig:B2B_converter} process all the output power of the generator and so it is employed with \acrshort{SCIG}, \acrshort{PMSM}, or \acrshort{WRSM}. \autoref{fig:B2B_converter_2} shows that it is also possible to reduce the amount of power that the power electronic has to be designed for, as done for example in the case of \acrshort{DFIG}, where only around 30\% of the rated power goes through the power electronic. Both these arrangements use IGBTs as switching devices.
\begin{figure}
  \begin{subfigure}{0.5\columnwidth}
    \centering
    \includegraphics[width=\columnwidth]{images/B2B_converter.jpg}
    \caption{Full scale BTB converter}
    \label{fig:B2B_converter}
  \end{subfigure}
  \begin{subfigure}{0.5\columnwidth}
    \centering
    \includegraphics[width=\columnwidth]{images/B2B_converter_2.jpg}
    \caption{BTB converter for a reduced amount of power}
    \label{fig:B2B_converter_2}
  \end{subfigure}
    \caption{2 configurations of BTB converters, source \cite{Olimpo_Anaya‐Lara}}
  \label{fig:B2B_converters}
\end{figure}

Medium voltage converters are more suitable for power ratings above 3 $\mesunt{\mega \watt}$, according to \cite{Olimpo_Anaya‐Lara}. The same source reports a switching configurations that can eliminate the wind turbine step-up transformer, making the employment favorable for turbine placed near the substation. \\
Alongside with the previously discussed configurations, \cite{Olimpo_Anaya‐Lara} reports that on generator side the \acrshort{VSR} can be replaced by a diode bridge rectifier, which is a passive converter. In this basic configuration the DC link voltage cannot be controlled and also the maximum power extraction is not possible. A solution to this drawback is the use of a boost converter between the diode rectifier and the DC link. An example of this configuration is shown in \autoref{fig:boost_converter}.
\begin{figure}[htb]
  \centering
  \includegraphics[width=0.5\textwidth]{images/boost_converter.jpg}
  \caption{Boost converter configuration, source \cite{Olimpo_Anaya‐Lara}}
  \label{fig:boost_converter}
\end{figure}

\subsection{Rotor braking system}
It is generally known that the lifespan of a wind turbine is in the order of 20 years, reaching even 30-35 with proper maintenance and repairing, as stated for example by the WT manufacturer Gamesa \cite{gamesa_life_turbine}. In all this long period of time the turbine has to be stopped in case of extreme events or maintenance, and so a rotor braking system is necessary.\\
An aerodynamics based and a mechanical emergency brake are usually employed. The first one acts by pitching the entire blades of 90 $\si{\degree}$ upwind (if the blade can be pitched from the root) or turning only the tip (whenever the pitching mechanism is not present, such as the old stall-regulated turbines). The generated aero torque slows down the rotor in few rotations at the most \cite{brake_dromstorre}. Usually these mechanism are spring operated in order to work also under electrical or hydraulic power failures.\\
The mechanical brake is mainly used as parking system and emergency brake \cite{brake_dromstorre}. This brake is usually placed on the high speed shaft (in the geared turbines) and activated by an hydraulic circuit \cite{brake_wiley}. 

\subsection{Yaw mechanism and yaw braking}
The yaw mechanism is the mechanical part allowing the nacelle and the rotor to be rotated around the tower axis. This feature is necessary for controlling the swept area of the turbine, which can be used either for maximizing the power production and also for reducing the power and loads when necessary. Since the moment of inertia of the nacelle-rotor assemble with respect yaw is high, then this approach is not used as primary power control method, even though some attempts are cited in \cite{Kim_2014}. On the other hand, yaw control can be exploited in wind parks in order to deflect the wake flow of one turbine providing to the downstream one an higher incoming wind speed, and thus an higher energy production \cite{Kim_2014}.\\
An example of yawing/brake mechanism assemble can be seen in \autoref{fig:fig_brake}. Other different arrangements are also possible, depending also on other factors such as the accessibility for installation and maintenance, the weight reduction, and the stress distribution. Furthermore the visible electric motors can be replaced by hydraulic ones \cite{Kim_2014}.\\
For what concerns yaw brake, it is often hydraulic regulated even tough again some electric configurations are available \cite{Kim_2014}. 
\begin{figure}
  \centering
  \includegraphics[width=0.5\textwidth]{images/fig_brake.jpg}
  \caption{Example of yawing mechanism, source \cite{Kim_2014}}
  \label{fig:fig_brake}
\end{figure}
