\section{Description of the most important subsystems} \label{sec:c_modelling_of_subsystem}
A general schema of a WT energy conversion chain is reported in \autoref{fig:fig_drivetrain} (source \cite{Olimpo_Anaya‐Lara}). 
\begin{figure}[htb]
  \centering
  \includegraphics[width=0.5\textwidth]{images/fig_drivetrain.jpg}
  \caption{General structure of a wind turbine drive train (source \cite{Olimpo_Anaya‐Lara})}
  \label{fig:fig_drivetrain}
\end{figure}
The incoming wind flow makes the blades rotate at a reasonably low rotational speed (let's say around 10 $\left[rpm\right]$ to give an order of magnitude). This rotation is then transmitted to the generator with or without the use of a step up gearbox, according to the requirement of the electrical machine. Finally, a power electronic converter conditions the power before the injection to the grid. Agin, the function and structure of this component depends on the employed generator type.\\
Conceptually on top of the physical components there are different hierarchical levels of control, that may have different objectives according to \cite{Olimpo_Anaya‐Lara}. The highest level supervise operator commands, faults, extreme events, startup and shutdown, possibly also coordinating with other turbines in the farm. Then a lower level is primary responsible of keeping the rotational speed within limits, and secondly reducing the aerodynamic loads, structural vibrations, and provide ancillary services to the grid.  The two most important controllers for the seak of this thesis are the one regulating the inclination of the blades (called pitch) and the electrical machine.\\
In the following subsections the most important subsystems will be described in order to provide some information about their structures and functionalities. These \textit{subsystems} are not only seen from a physical point of view but also as functional blocks interconnected to provide to the turbine its full capabilities.

\subsection{Basic aerodynamic model for the blade}
\subsubsection{Mechanical power and power coefficient}\label{subsec:mech_pow_and_pow_coeff}
The mechanical power that a \acrshort{WT} can extract from the wind is expressed by the well known equation:
\begin{gather}
    P = \frac{1}{2}\ A \, \rho \, c_P \, V_0^3 \ \ \left[\si{\watt}\right] \label{eq:power} \\
    V_0 = \sqrt[3]{\frac{2 \ P}{A \, \rho \, c_P}} \ \ \left[\si{\meter \per \second}\right] 
\end{gather}
where A is the rotor swept area $\left[\si{\square \meter}\right]$, $\rho$ is the air density in $\left[\si{\kilo\gram\per\cubic\meter}\right]$, and $c_P$ is the power coefficient. 

The \acrfull{cp} is defined as the ratio between the power extracted by the \acrshort{WT} divided by the available from the source one:
\begin{equation}
    c_P = \frac{P_{\text{extracted}}}{P_{\text{available}}} \ \ \left[-\right]
    \label{eq:c_P}
\end{equation}
and it is computed from the analysis of the elementary blade cross section. More in detail, the $c_P$ is a function of the \acrfull{TSR} $\lambda$ and the pitch angle $\theta$. The \acrshort{TSR} is defined as:
\begin{equation}
    \lambda = \frac{\omega R}{V_0} \ \ \left[-\right]
    \label{eq:TSR}
\end{equation}
with $\omega$ being the rotor rotational speed in $\left[\si{\radian \per \second}\right]$, R is the rotor radius in $\left[\si{\meter}\right]$, and $V_0$ is the incoming wind speed in  $\left[\si{\meter \per \second}\right]$. The pitch angle is the rotation angle of the blade around its main axis, and by adjusting it the \acrfull{AOA} of the entire blade changes, producing a modification of the harvested power. The \acrshort{AOA} is the one between the incoming wind flow and blade's chordline, denoted as $\alpha$ in \autoref{fig:velocity_triangle}. 
\begin{figure}[htb]
    \centering
    \includegraphics[width=0.5\textwidth]{images/velocity_triangle.png}
    \caption{Airfoil cross-section of chord length $c$. The Angle of attack $\alpha$ is the one between the incoming \acrshort{WS} $V_0$ and the chordline. $D$ and $L$ are the produced drag and lift forces. F and M are the total force and torque on the cross section. Source of the figure: \cite{Aerodynamics_of_wind_turbines}}
    \label{fig:velocity_triangle}
\end{figure}

Usually the power is reduced by decreasing the \acrshort{AOA} by pitching the leading edge of the blades up against the wind, but also the opposite is possible \cite{Aerodynamics_of_wind_turbines}. The first strategy is called feathering while the second stalling.

\subsubsection{Thrust force and thrust coefficient}
The thrust force, usually denoted as T, is the force produced by the rotor in the streamwise direction. It produces the wind speed reduction from the incoming one to the one in the wake, and it is compute as:
\begin{equation}
    T = \frac{1}{2} \, A \, \rho \, c_T \, V_0^2 \ \ \left[\si{\newton}\right]
    \label{eq:thrust_coeff}
\end{equation}
The thrust coefficient $c_T$ is obtained similarly to the power one, as it will be shown later in \autoref{subsec:BEM_algorithm}.

\subsubsection{BEM algorithm}\label{subsec:BEM_algorithm}
The relationship between the \acrshort{cp}, the \acrshort{TSR} and the pitch angle is not linear, and involves the aerodynamic interaction between the incoming wind and the airfoil characteristics. Different methods are available for computing this interactions, with different levels of complexity and fidelity. A common analytical mathematical tool is the so called \acrfull{BEM}, solving the problem by providing the tangential and normal forces to the blade. The assumptions on which it is based are the absence of radial dependency of the blade-wind interaction and a constant force from the blades on the flow. The last assumption corresponds to having a rotor with an infinite number of blades, and so it is later corrected with a coefficient taking into account their finite number \cite{Aerodynamics_of_wind_turbines}.\\
At the basis of the computations done in this work there is a \acrshort{BEM} algorithm implemented in Matlab whose pseudo-code is reported in \autoref{app:BEM_code}. In particular, \autoref{alg:main_BEM} reports the main functions that loops between the different TSR, pitch, and cross sections, while \autoref{alg:detailed_BEM} reports the implementation details leading to the computation of the local (i.e. for a cross section) thrust and power coefficients. A list of the main terms used in the code is reported in \autoref{tab:BEM_code_notation}. \\
Other methods that may be cited but not further discussed because not used here are the vortex theory, 2D/3D actuator line theory, 2D/3D full CFD models.

\subsection{Drivetrain}
The term \textit{drivetrain} identifies all the components of the mechanical system used to transfer the power from the rotor to the generator.
The classical components of a drivetrain are the \textit{main shaft} (i.e. the shaft connected to the rotor) which usually rotates at low speed, the \textit{gearbox} (which is not always present), and the generator. According to \cite{Olimpo_Anaya‐Lara}, in 85\% of the drivetrains a gearbox is present but this value may be biased by the older installed machines. Its primary use is to step up the rotational speed, and so to use high speed machines. According to the gearbox ratio, they may be divided between \textit{medium speed} (around 1/10) and \textit{high speed} (which can reach the range 1/90-1/120).\\
The other possibility is to have a \textit{gearless} configuration, and the direct drive is the most common, while hydraulic transmission are under development, \cite{Olimpo_Anaya‐Lara}.\\
The geared concept is the older one since it has been available since 1970s, and it amplifies the inertia of the generator by the square of the inverse of the gear ratio, reducing the torque variations sensitivity. The gear-less concept was introduced in 1991, it requires a lower number of components, eliminates transmission losses and gearbox failure. On the other hand the electrical machine has higher dimensions and weights because more space for the poles is necessary.\\
In the market it is not clear which of the two solutions will be the dominant, since both are available. \autoref{tab:transmission_review} summarises some interesting features of multi mega-watt \acrshort{WT} produced by some of the most important manufacturer currently in the market. 

\begin{table}[htp]\centering
    \caption{Review of the drivetrain characteristic of some WTs available on the market. The most used electrical machines are the \acrfull{PMSM} and the \acrfull{DFAG}.} \label{tab:transmission_review}
    \scriptsize
\begin{tabular}{lcccccc}\toprule
Manufacturer &Model &Power [MW] &Location &Year &Transmission &Electrical Machine \\\midrule
\multirow{5}{*}{Siemens Gamesa} &SG 6.6-170 &6.6 &Onshore &2021 &Geared & \\
&5.X platform &5.6/7.0 &Onshore &2021 &Geared & \\
&SG 8.0-167 DD &8.0 &Offshore &2019 &Direct Drive &PMSM \\
&SG 11.0-200 DD &11.0 &Offshore &2022 &Direct Drive &PMSM \\
&SG 14-222 DD &14.0 &Offshore &2024 &Direct Drive &PMSM \\ \midrule
\multirow{5}{*}{Vestas} &V117-4,2 MW &4.2 &Onshore & & Geared &PMSM \\
&V164-9.5MW &9.5 &Offshore & &Medium speed &PMSM \\
&V164-10MW &10.0 &Offshore & &Medium speed &PMSM \\
&V174-9.5MW &9.5 &Offshore & &Medium speed &PMSM \\
&V236-15.0MW &15.0 &Offshore & &3 planetary stages &PMSM \\ \midrule
\multirow{2}{*}{General Electric} &Haliade 150-6MW &6.0 &Offshore &2016 &Direct Drive &PMSM \\
&Haliade 150-12MW &12.0 &Offshore &2019 &Direct Drive &PMSM \\\midrule
\multirow{2}{*}{Nordex} &N175/6.X &6.0 &Onshore & &High speed &DFAG \\
&N163/5.X &5.0 &Onshore & &High speed &DFAG \\ \midrule
\multirow{4}{*}{Mingyang} &MY2.0 &2.0 &Offshore & &Gearbox &DFAG \\
&MYSE3.0MW &2.6/3.6 &Offshore & &Gearbox &PMSM \\
&MYSE4.0MW &4.0/5.0 & & &Gearbox &PMSM \\
&MYSE6.0MW &5.5/8.3 & & &Gearbox &PMSM \\
\bottomrule
\end{tabular}

\end{table}

It is possible to notice that Siemens-Gamesa and General Electric use direct drive concepts, while Vestas still prefers a geared solution even tough with the same kind of electrical machine.

\subsection{Electrical generator}
Two kinds of electrical machine may be used: the permanent magnet or the electrically excited. The popularity of one or the other depends also on the cost of the magnets \cite{1-s2.0-S0040162519313691-main}. \\
\textcolor{red}{Write something about the used generator in the different types}

\subsection{Other subsystems}
\textcolor{blue}{Could it be a good idea to add something about the pitching mechanism, the mechanical brake, and the power electronic processing the output of the generator even though they are not implemented in the simulator?}
