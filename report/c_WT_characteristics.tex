\newpage
\section{Introduction to the Wind Turbines}\label{sec:c_WT_characteristics}
In this section the historical evolution of the wind energy converters will be briefly summarized, then some data on the actual penetration of the resource in the european market will be presented as long as some points for the future development. \\
The analysis of the data will be focused on the european area, mainly because this where the writer lives, but also because in the continent there is a growing interest in the renewable in order to fulfill the target of reducing the greenhouse emission of 55\% by 2030 wrt the 1990 level. 

\subsection{Historical evolution}
Going back in the past it could be seen that human has \textit{always} used the wind energy resource, for example ships were driven by wind along the Nile River as early as 5000 BC.\\
Windmills can be considered as grandparents of modern wind turbines since in them the wind resource were harvested in a similar way. Initially, the were used for directly pumping waters or grinding grains but later on, at the beginning of the XX century, the electricity become more popular and so the windmills rotor were connected to electrical generators, starting the history of wind turbines as we know them today. Poul la Cour can be mentioned as one of the earlier contributors of this new technology since he was one among the first to connect a generator to a turbine \cite{Aerodynamics_of_wind_turbines}. \\
The first upwind three-bladed, stall regulated rotor, connected to an AC asynchronous generator running at almost constant speed (known as the \textit{Danish concept}) was introduced in mid-1950s. Later on, after the oil crisis of the 1973, more nations started to be interested in the possibility of using the wind resource to be less dependent from oil import and from this moment on, the wind energy industry as become more and more important. 

\subsection{Some data on the current employment and state of the art of the resource}
In 2022 Europe commissioned 19 GW of new wind plants (4\% more than in the 2021) reaching a totoal amount of 255 GW of installed capacity, according to \cite{wind_europe_data_2022}. This amount covered 17\% of the EU-27+UK electrical demand in 2022, with Denmark and Ireland the countries where the share of wind energy in the electrical mix was the higher (55\% and 34\% respectively). The amount of electricity from wind in the italian mix was 7\% \cite{wind_europe_data_2022}.\\
Even tough onshore plants were 87\% of new installation in Europe in 2022 \cite{wind_europe_data_2022}, an increasing trend is the deployment of the turbine in large \acrfull{OWF}, because there are some advantages in terms of more space available for bigger plants, steadier wind speeds with less turbulence, higher capacity factors (30-45\% for onshore and around 50\% offshore according to \cite{wind_europe_data_2022}), and lower visual and acoustic impacts \cite{current_staus_and_future_trends_of_offshore_wind_power_in_europe}. The principal drawbacks of this solution are the more difficult access to the sea, the cost of the marine foundations, the connection to the grid through marine cables.  \\
Talking about power, onshore installed turbines in 2022 had an average power rating of 4.1 MW, while the offshore ones of 8.0 MW. Furthermore the average power for new ordered turbine increased up to 5.1 MW while the ordered offshore turbines reached a record high, averaging 12.2 MW, thanks to which the produced power is expected to increase from 2023 on.\\

From the technical point of view, the most common employed families of generator are the doubly fed induction, the squirrel cage induction and the permanent magnet synchronous.\\
In terms of power transmission, high voltage alternate current is preferred over direct current for plants closer than 50 km from the shoreline, while the other way around for further plants. The last does not require reactive power compensation but typically have higher costs \cite{current_staus_and_future_trends_of_offshore_wind_power_in_europe}.\\
\cite{current_staus_and_future_trends_of_offshore_wind_power_in_europe} reports that up 2019 in Europe 110 offshore wind energy farm with an installed power of more than 150 MW were installed. The countries that contribute to this grown the most are UK and Germany, while Belgium, Denmark and the Netherlands covers the 20\% of new installations.

\subsection{Future development}
The natural path of turbine manufactories is the development of larger turbines in the next years. In general, the increase of the size increases the cost but since less turbine will be necessary, then the number of units needed will be reduced \cite{current_staus_and_future_trends_of_offshore_wind_power_in_europe}.\\
There is not a wide agreement about the reachable size of wind turbines, however the upscaling of some components could not be done with only a geometrical increasing but requires some changes in the fundamental technology. For example, scaling the blades of the same quantity in all the directions does not increase the aerodynamic stress on them since both their load and their resistance scales with the same ratio. Unfortunately, at the same time the gravitational load increases with the cubic of the scaling ratio meaning that the blade is subject to an higher centrifugal load that is not sustained from the blade itself as it is. \\
\cite{current_staus_and_future_trends_of_offshore_wind_power_in_europe} suggests that generators in the size 6-12.5 MW will be common in OWF by 2025 and 20 MW by 2030.\\
In the period 2023-2027, Europe expect to install 129 GW of new wind power capacity, with 74\% of these sited onshore. These installation schedule is lower than the EU needs in order to meet the REPowerEU goals by 2030 (31 GW/year) according to \cite{wind_europe_data_2022}. \\
In the period 2023-2027 it will be also expected to start decommissioning old plants. Some of them will be definitely removed from the system, while others will be repowered. This operation will on average trebles the output while reduces the number of turbine by 25\%. \\
An important aspect to be taken into account for the development are also regulations and permitting. For example the first italian offshore plant was proposed back in 2008, but was commissioned only in the 2022 due to delays in the authorization process \cite{il_post} (in italian). In such a large amount of time the technology increases the size of the generators and there may be the risk that what was initially designed becomes outdate even before being deployed. To overcome similar problems, the EU member states agreed in speed up the the permitting procedures and the european commission has also suggested new laws and guidance to simplify the permitting of wind energy projects. It is expected that these action will release the 47 GW onshore wind capacity that was stuck in administrative procedure during 2022 (source \cite{wind_europe_data_2022}).  