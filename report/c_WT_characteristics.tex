\newpage
\section{Introduction to the Wind Turbines}\label{sec:c_WT_characteristics}
\subsection{State of the art of the WT technologies}
\subsection{General structure of the energy conversion chain}
A general schema of a WT energy conversion chain is reported in \autoref{fig:fig_drivetrain} (source \cite{Olimpo_Anaya‐Lara}). \\
\begin{figure}[htb]
  \centering
  \includegraphics[width=0.5\textwidth]{images/fig_drivetrain.jpg}
  \caption{General structure of a wind turbine drive train (source \cite{Olimpo_Anaya‐Lara})}
  \label{fig:fig_drivetrain}
\end{figure}
The incoming wind power makes the blades rotate at a reasonably low rotational speed (around 10 $\left[rpm\right]$ to give an order of magnitude). This rotation is then transmitted to the generator with or without a step up gearbox, according to the requirement of the electrical machine. Finally, a power electronic converter conditions the power before the injection to the grid. Agin, the function and structure of this component depends on the employed generator type.

\subsection{Operational regions of a WT}
Even thought the scope of a \acrlong{WT} is usually to harvest as much mechanical energy as possible by reducing the wind's kinetic one, different working regions and corresponding wind speeds levels may be identified depending on the followed literature. For example \cite{Olimpo_Anaya‐Lara} uses two regions (called I and II), \cite{5874598} uses three (called 1, 2, 3), while \cite{10-MW_Direct-Drive_PMSG-Based_Wind_Energy_Conversion_System_Model} uses five (called I, II, III, IV, and V). A common denominator is that in all of them the \textit{partial load} and the \textit{full load} regions are identified. The \textit{partial load} is bounded between the \textit{cut-in wind speed} (i.e. the minimum wind speed required to enabling the blade rotation) and the \textit{rated wind speed} (i.e. the lowest velocity at which the rated mechanical power is produced), and the mechanical power is below the rated one. Complementary, in the \textit{full load} the mechanical power would exceed the rated one, but it is limited by the controller. The boundaries of this regime are the rated wind speed and the \textit{cut-out} one (i.e. the maximum one taken into account during the design phase). \\
The further identified region in \cite{5874598} and region I in \cite{10-MW_Direct-Drive_PMSG-Based_Wind_Energy_Conversion_System_Model} is the one below cut-in, when the rotor is kept still by the brakes. Region III in \cite{10-MW_Direct-Drive_PMSG-Based_Wind_Energy_Conversion_System_Model} is the switching between the partial and full load, while IV is the region above cut-out when the rotor is kept locked by the brakes. Even though the division in 5 zones is more complete, the one with 2 is sufficient for identifying the conceptual differences. 
\begin{figure}[htb]
    \centering
    \includegraphics[width=0.5\textwidth]{images/operating_reagions.png}
    \caption{Wind speed division in 5 different operating regions according to \cite{10-MW_Direct-Drive_PMSG-Based_Wind_Energy_Conversion_System_Model}}
    \label{fig:operating_reagions}
\end{figure}
