\begin{abstract}
  Wind is one of the most used renewable energy source, and it is playing a key role in achieving the European target of renewable energy production. Wind turbines are articulated devices, composed of many subsystems of different domains such as aerodynamic, mechanical, and electrical. The control of these devices in order to let them work in a specific configuration can be quite complex.\\
  The aim of this thesis is to develop a simulator of a wind turbine generator, from the resource up to the electric machine, to be used for testing different controllers for the blade pitching mechanism and the electric generator. Then, after having validate the functionality of the emulator on some simple cases, it has been used to test different control laws. Initially, the objective has been set to the maximization of the power extracted from the resource, then on the maximization of the electrical power at the output terminal of the generator, and finally a control law taking into account the uncertainty on the knowledge of the physical parameters has been tested.\\
  \textcolor{red}{Write some conclusions once they will be ready}

\end{abstract}