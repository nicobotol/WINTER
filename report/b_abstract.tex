\begin{abstract}
  Wind is one of the most technically relevant renewable energy resources, and it is playing a key role in achieving the European targets on renewable energy production. At the same time, wind turbines are complex devices, composed of many subsystems spanning domains such as aerodynamics, mechanics, and electrical machines, hence their control can be quite complex.\\
The aim of this thesis is to develop a software emulator of a wind turbine, from the resource up to the electric machine, with a particular focus on the controller of the electrical machine and the pitching mechanism. After having validated its functionality on some simple cases, the emulator is used to test different control laws. The objective is initially set on the maximization of the power extracted from the resource, then on the maximization of the electrical power at the output terminal of the generator, and finally on making the control law robust against uncertainty in the knowledge of the physical parameters.\\
The results of the simulations performed on the emulator provide preliminary validation of the proposed laws.
\end{abstract}